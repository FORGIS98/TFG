\documentclass{article}
\usepackage[utf8]{inputenc}

\begin{document}

\textbf{Nombre completo del alumno:} Jorge Sol González \\
\textbf{Número de matrícula:} Z170212 \\
\textbf{Grado en:} Ingeniería Informática \\
\textbf{Nombre del tutor académico:} Francisco Javier Soriano Camino \\
\textbf{Nombre del trabajo:} “Diseño e implementación de un SDK Android para facilitar la interacción de aplicaciones móviles con una Blockchain ” \\

\vspace{1cm}
\textbf{Resumen general del trabajo:}
En el contexto del Campus Blockchain de la Cátedra INETUM de Negocio TI y Transformación Digital se está desarrollando un proyecto para registrar evidencias inmutables de diferentes hitos en un itinerario formativo con el objetivo de poder acreditar de forma certificada las competencias adquiridas en el mismo. Para ello se hace uso de la tecnología Blockchain. Como primer producto mínimo viable (MVP), se plantea un caso de uso en el ámbito académico: la asistencia de estudiantes a congresos, prácticas o actividades de diferente índole en el ámbito académico en una red blockchain privada tipo Ethereum (Quorum). Además del despliegue de infraestructura y la configuración de la red, se plantea el desarrollo de una serie de Smart Contracts y unas APIs para facilitar la integración con las diferentes aplicaciones que intervienen en los procesos actuales.  En ese contexto, el presente Trabajo de Fin de Grado consistirá en la implementación de una aplicación móvil para los estudiantes capaz de invocar estas APIs o, directamente, los Smart Contracts. Esta aplicación se diseñará de forma totalmente modular, permitiendo el incremento de funcionalidad de forma sencilla. Además, para poder reutilizar la funcionalidad más específica de custodia de claves, firma de transacciones e interacciones con la Blockchain en otras aplicaciones, esta parte se componetizará en SDKs.

\vspace{1cm}
\textbf{Lista de objetivos concretos del trabajo:}
\begin{enumerate}
  \item Estudio de métodos de firma de transacciones Blockchain
  \item Análisis, diseño e implementación de almacenamiento de claves en dispositivos móviles
  \item Diseño de la arquitectura modular del aplicativo móvil
  \item Implementación de la aplicación móvil y SDKs
  \item Redacción de una guía de buenas prácticas
\end{enumerate}

\vspace{1cm}
\textbf{Desglose de la dedicación total del trabajo en horas:}
324 horas distribuidas de la siguiente manera:

\begin{itemize}
  \item Estudio y análisis del estado del arte y de las tecnologías a utilizar (29 horas).
  \item Análisis y diseño de un SDK Android (30 horas).
  \item Implementación del SDK para Android (100 horas).
  \item Diseño e implementación de un conjunto de pruebas que comprueben el correcto funcionamiento del SDK de forma automática (60 horas).
  \item Elaboración de una documentación del SDK desarrollado (10 horas).
  \item Diseño e implementación de una aplicación Android a modo de prueba de concepto que integre el SDK desarrollado (50 horas).
  \item Elaboración de una memoria de TFG que recoja todo el trabajo realizado (35 horas).
  \item Elaboración de una presentación PPT que sintetice y presente de forma clara los aspectos esenciales del trabajo realizado por el alumno (10 horas).
\end{itemize}

\vspace{1cm}
\textbf{Lista de conocimientos previos recomendados para realizar el trabajo:}
\begin{itemize}
  \item Blockchain (Ethereum, Quorum)
  \item Docker
  \item Android
  \item Git
\end{itemize}

\end{document}
