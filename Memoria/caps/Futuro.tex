\chapter{A Futuro}
\label{cap:Futuro}

Actualmente la aplicación ``Estublock'' puede utilizarse para registrar las asistencias a eventos en una universidad sin problema. Sin embargo es largo el camino que le queda para poder decir que esta terminada y que se pueda lanzar la primera versión de la aplicación. Ahora mismo es lo que se conoce como ``una versión alpha''. Esta listo para ser probado pero no para que se utilice de forma masiva. ¿Que avances y siguientes pasos necesita la aplicación para poder lanzar la primera versión \emph{V.1.0.0}? \\

Actualmente el código es ligeramente caótico, durante los meses de desarrollo ha ido sufriendo muchos cambios, renombrado de archivos, creación y eliminación de código y archivos que en su momento tenían utilidad pero ya no. Y todo esto causa que tanto código como archivos, no estén tan ordenados y organizados como deberían. Lo ideal es entonces ordenarlo en carpetas con nombres significativos y eliminando toda repetición de código así como archivos que no hagan falta o crear archivos que pueden hacer que el código sea más mantenible. Esto es fundamental, pues para un futuro programador es más fácil coger un proyecto ordenado y estructurado siguiendo un ``modus operandi'', a coger un proyecto caótico. Además, ordenar el código reduce errores al minimizar las líneas de código (por quitar código repetido) y permite tener mayor control de las funciones que causan problemas. \\
 
Como toda pieza de software, se necesitan tests para probar el correcto funcionamiento de la aplicación. Con una buena batería de tests, el programador puede probar el código que escribe, reduciendo los errores y reduciendo el tiempo que se tarda en corregirlos. Esto se debe a que los errores se detectan con más antelación y de forma más concreta. Dando lugar a un código mas estable, escalable y sostenible en el tiempo. Como no se disponía de experiencia en el desarrollo de aplicaciones móvil, ha sido difícil crear tests unitarios pues el código sufría muchos cambios diariamente con muchas modificaciones en el nombre de variables. Ahora que el código es más estable, y se tiene más experiencia, es un buen momento para desarrollar estos tests unitarios. \\

La documentación es también uno de los pilares de un buen software. Actualmente las funciones tienen pequeños comentarios, excepto las del SDK que están completamente comentadas. Sería ideal añadir comentarios completos en el resto de funciones siguiendo el estilo de comentarios de javadocs para poder generar posteriormente con esta herramienta una documentación más detallada. Muchas partes del código se explican por si mismas, pero la documentación es fundamental para que un proyecto sea mantenido en el tiempo, principalmente para que los futuros desarrolladores entiendan el código ya existente y no pierdan tiempo en averiguar que es lo que hace el código. \\

Con respecto a la interfaz de usuario, esta se ha mantenido muy sencilla. Pero una mejora en el diseño o un remodelado visual pueden hacer que la aplicación sea mucho más atractiva, haciendo que los usuarios disfruten mucho más con su uso. Así como, mejorar la experiencia de los usuarios al utilizarla. Para ello, se puede por ejemplo añadir una opción que permita al usuario personalizar a su gusto el tema de la aplicación, o alguna pantalla o botón que quieran poder mover de sitio. \\

Como se ha mencionado antes, la aplicación permite actualmente registrar las asistencias a eventos, así como crear nuevos eventos, aunque estas sean sus funcionalidades principales, se puede explotar la aplicación mucho más añadiendo nuevas funcionalidades como permitir modificar los datos personales a los usuarios, listar los eventos a los que se ha asistido a lo largo del tiempo o ver los usuarios que han asistido a un evento concreto. También, muy importante, ha de hacerse la diferencia entre alumno y profesor, mostrando distintas pantallas y funcionalidades según el usuario sea un profesor o un alumno. \\

Por último, si miramos más en el futuro. Uno de los objetivos estrella del proyecto ``Estublock'' es llegar a todos los campus de la Universidad Politécnica de Madrid, y posteriormente, al resto de universidades de Madrid y España. Aunque sea un objetivo muy lejano, no hay olvidarlo, este proyecto puede tener un impacto muy positivo en las universidades. Por ello, aunque sea lejano, hay que tener en mente que queremos llegar a todo el mundo. 
