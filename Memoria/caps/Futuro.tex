\chapter{Futuro}
\label{cap:Futuro}

Actualemnte la aplicación puede usarse, sin embargo es largo el camino que le queda para poder decir que esta terminada y que se pueda lanzar la primera versión de la aplicación. Ahora mismo es lo que se conoce como ``una versión alpha''. Esta listo para ser probado pero no para que se utilice de forma masiva. ¿Que avances y siguientes pasos necesita la aplicación para poder lanzar la primera versión \emph{V.1.0.0}?

\begin{enumerate}
\item \textbf{Refactorización: } Actualmente el código es ligeramente caótico, lo ideal es entonces ordenarlo en carpetas con nombres significativos y eliminando toda repetición de código. Esto es fundamental, pues para un futuro programador es más fácil coger un proyecto ordenado y estructurado siguiendo un ``modus operandi'' a coger un proyecto caótico. Además, ordenar el código reduce errores al minimizar las líneas de código (por quitar código repetido) y permite tener mayor control de las funciones que causan problemas. 

\item \textbf{Tests: } Como toda pieza de software, se necesitan tests para probar el correcto funcionamiento de la aplicación. Con una buena batería de tests, el programador puede probar el código que escribe, reduciendo los errores y reduciendo el tiempo que se tarda en corregir errores. Esto se debe a que los errores se detectan con más antelación y de forma más concreta. Dando lugar a un código mas estable, escalable y sostenible en el tiempo. 

\item \textbf{Documentación: } Actualmente las funciones tienen pequeños comentarios, sería ideal añadir comentarios completos siguiendo el estilo de comentarios de javadocs para poder generar posteriormente con javadocs una documentación más detallada. Muchas partes del código se explican por si mismas, pero la documentación es fundamental para que un proyecto sea mantenido en el tiempo.

\item \textbf{Experiencia de Usuario: } 
  \begin{itemize}
  \item Mejorar el diseño de la aplicación móvil, así el usuario tiene una experiencia más agradable al utilizar una aplicación visualmente mas atrayente. Por todos es sabido, que una aplicación fea atrae menos que una aplicación bonita y cuidada. 
  \item Permitir al usuario editar el tema de colores de la aplicación. Esto permite al usuario darle un toque personal a la aplicación.
  \end{itemize}

\item \textbf{Funcionalidades: } Añadir más funcionalidades. Primero habría que ver cuales, diseñar un poco las pantallas y luego desarrollarlas. Ideas disponibles actualmente son: 
  \begin{itemize}
  \item Permitir modificar datos personales.
  \item Listar los eventos a los que ha asistido un usuario.
  \item Listar los usuarios que han asistido a un evento (para los profesores).
  \end{itemize}
\end{enumerate}

Me ha gustado mucho desarrollar este proyecto, al haberlo hecho desde cero es algo muy personal y seguiré desarrollandolo con el paso del tiempo, así como hacer pública el SDK para que otras personas puedan contribuir al mismo y con suerte poder poner mi granito de arena en la aplicaciones móvil. 
