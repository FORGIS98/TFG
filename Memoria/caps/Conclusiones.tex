\chapter{Conclusiones}
\label{cap:Conclusiones}

La tecnología blockchain ha venido para quedarse, cada día que pasa se utiliza más, más empresas toman la decisión de utilizar su potencial y cada vez más la población conoce su existencia. La popularidad de blockchain se debe a que es un sistema seguro, fiable, transparente, descentralizado y flexible. Permite hacer transacciones bancarias, validar la identidad de un usuario, registrar la reserva de un billete de avión\dots \\

Para explotar esta tecnología y su creciente popularidad, han nacido aplicaciones web y móvil para que las personas puedan aprovechar las redes blockchain. Tenemos por ejemplo la aplicación móvil \emph{Guts} para la compra y venta de entradas a eventos (cine, conciertos, teatro\dots). \emph{LifeID}, plataforma desde la que gestionar tu identidad digital con total seguridad o \emph{Voatz} que permite participar en votaciones a la gente desde sus dispositivos móviles, votaciones importantes como votar al presidente de un país. \\

Despues de estudiar a fondo la tecnología blockchain, creo que tiene aún muchísimo que ofrecer, y se usará con seguridad por los gobiernos de todos los países una vez sea más estable. Blockchain puede tener un impacto a nivel económico enorme, pero también, puede mejorar la trazabilidad de los datos, la fiabilidad de los datos y la transparencia de todo, desde sanidad, servicios, bienes materiales\dots Sin embargo, aunque la tecnología mejora día a día, sigue siendo bastante nueva y puede presentar algunos fallos, lo que la hace menos robusta de lo deseado para poder implementarla de forma masiva. \\

Finalmente, despues de aprender desde cero a desarrollar mi primera aplicación Android, y aprender a hacer llamadas desde esta a la red blockchain, creo que la implementación de las librerias disponibles deberían tener más en cuenta a los dispositivos móvil, las librerías que he probado funcionan muy bien, pero no son específicas para aplicaciones Android. Esto hace que algunas funciones sean lentas, pesadas y que requieran de una potencia de cálculo que un móvil puede no tener. Sin embargo, sí se esta mejorando poco a poco, y cada vez hay más soporte de librerías para Android específicamente. \\

Para terminar, el tiempo realizando este trabajo de fin de grado ha pasado volando, y esto solo puede significar una cosa, he disfrutado mucho, he aprendido mucho, y he mejorado mucho. Además, no solo me ha gustado, sino que espero con ansia que la aplicación Estublock y el SDK sigan siendo desarrollados y sigan creciendo, y deseo ver en un par de años este proyecto en uso por alumnos de varias universidades de Madrid. Con este TFG, a parte de ver mejoradas mis habilidades programando en varios lenguajes, utilizando varias librerias y solventando muchos problemas, también he tenido que aprender a diseñar un proyecto, diseñar un SDK, estructurar y diseñar las pantallas poniendome en la piel del usuario final. He redescubierto la metodología SCRUM, he descubierto herramientas como \textit{taskwarrior} y organización del trabajo con \textit{timeboxing}, y muchas más cosas que he usado y aprendido de forma inconsciente. \\

Dicho de otro modo, este TFG marca un antes y un después en mi vida como desarrollador, sintiendome ahora capaz de enfrentarme a cualquier proyecto de la vida. 
