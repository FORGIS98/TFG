\chapter{Conclusiones}
\label{cap:Conclusiones}

La tecnología blockchain ha venido para quedarse, cada día más empresas toman la decisión de utilizar su potencial y cada vez más, la población sabe de su existencia. La popularidad de blockchain se debe a que es un sistema seguro, fiable, transparente, descentralizado y flexible. Permite hacer transacciones bancarias, validar la identidad de un usuario, registrar la reserva de un billete de avión\dots Para explotar esta tecnología y su creciente popularidad, han nacido aplicaciones web y móvil para que las personas puedan aprovechar las redes blockchain. Tenemos por ejemplo la aplicación móvil \emph{Guts} para la compra y venta de entradas a eventos (cine, conciertos, teatro\dots). \emph{LifeID}, plataforma desde la que gestionar tu identidad digital con total seguridad o \emph{Voatz} que permite participar en votaciones a la gente desde sus dispositivos móviles, votaciones importantes como votar al presidente de un país. \\

Después de aprender desde cero a desarrollar mi primera aplicación Android, y aprender a hacer llamadas desde esta a la red blockchain, he echado en falta un mayor soporte de las librería que he utilizado y que he encontrado disponibles a dispositivos Android. Las librerías que he probado funcionan muy bien, pero no son específicas para aplicaciones Android. Esto hace que algunas funciones sean lentas, pesadas y que requieran de una potencia de cálculo que un móvil puede no tener. Por ejemplo, las llamadas a redes blockchain han de hacerse en un Thread diferente del principal para evitar que la pantalla de la aplicación se congele. El problema con Android, es que aunque este programado en Java, una librería de Java como web3j (la usada en este proyecto para comunicar con la red blockchain) no esta optimizada para Android, simplemente funciona bien con Java, pero hay que retocarla un poco para que funciona bien en Android. Por esta misma razón, el SDK que se ha implementado sí tiene en cuenta esta necesidad de enfocarse a aplicaciones Android. \\

Durante el desarrollo de este TFG se ha tenido la oportunidad de poner en práctica el funcionamiento de la red blockchain levantada para la aplicación móvil en el registro de asistencias del congreso anual TryIT que se celebra en nuestra facultad, así como en el taller de machine learning que se impartió en la Cátedra Inetum. En ambos casos, la red blockchain ha respondido bien, registrando perfectamente las asistencias. Esto es muy bueno, pues las mismas llamadas utilizadas para estos eventos, son las que el dispositivo móvil realiza. Esto junto con las pruebas realizadas para comprobar el correcto funcionamiento de la aplicación móvil, da pie a confirmar que se ha cumplido con el objetivo principal de este trabajo. Este es, registrar las asistencias a eventos con una aplicación móvil guardando la información en la red blockchain. Y para esto último desarrollar un SDK que también se ha logrado terminar con exito. Lógicamente a la aplicación le queda mucho por crecer y avanzar, se verá mas a fondo en el apartado \hyperref[cap:Futuro]{A Futuro}. \\

Personalmente, el tiempo realizando este trabajo de fin de grado ha pasado volando, esto solo puede significar una cosa, he disfrutado mucho, he aprendido mucho, y he mejorado mucho. Además, no solo me ha gustado, sino que espero con ansia que la aplicación Estublock y el SDK sigan siendo desarrollados y sigan creciendo, y deseo ver en un par de años este proyecto en uso por alumnos de varias universidades de Madrid. Con este TFG, a parte de ver mejoradas mis habilidades programando en varios lenguajes, utilizando varias librerías y solventando muchos problemas, también he tenido que aprender a diseñar un proyecto, diseñar un SDK, estructurar y diseñar las pantallas poniéndome en la piel del usuario final. He redescubierto la metodología SCRUM, he descubierto herramientas de gestión de tareas como \textit{taskwarrior} y gestión del tiempo con \textit{timeboxing}, y muchas más cosas que he usado y aprendido de forma inconsciente. Dicho de otro modo, este TFG marca un antes y un después en mi vida como desarrollador, pues ahora me veo capaz de enfrentarme a cualquier proyecto, pues por duro que sea, se que mis ganas de aprender y mi curiosidad me harán seguir adelante. 
