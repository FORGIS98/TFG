\chapter{Introducción}
\label{cap:Introduccion}

En esta sección se presenta el contexto del TFG, la motivación detrás de este trabajo, los objetivos a cumplir y una breve explicación de la estructura del documento.

\section{Contexto}

A lo largo de estos dos últimos años, hemos tenido la suerte de estar trabajando en la \emph{Cátedra Inetum} de la escuela técnica superior de ingenieros informáticos con la colaboración de la empresa Inetum. Inetum es una compañía de servicios ágil que proporciona servicios y soluciones digitales y un grupo global que ayuda a compañías e instituciones a aprovechar al máximo la corriente digital. Durante estos dos años, hemos estado investigando sobre la tecnología blockchain y el potencial que puede aportar a las personas, en concreto enfocado al mundo universitario. \\

Dentro de esta cátedra, y con los conocimientos e investigaciones realizadas, se ha ideado desde cero un proyecto muy innovador. ``Estublock'', surge ante la necesidad de un sistema fiable, robusto y rápido, de registro de asistencias a exámenes. A partir de esta necesidad, se ha expandido la funcionalidad de la aplicación para cubrir otros tipos de eventos como prácticas, talleres, laboratorios\dots Y la meta, es hacer de Estublock él sistema por defecto para validar la asistencia a exámenes, prácticas, talleres\dots de todas las facultades de la Universidad Politécnica de Madrid, y expandir al resto de universidades públicas. 

\section{Motivación}

  Por suerte o por desgracia, es habitual que unos pocos universitarios presenten quejas todos los años contra algún profesor, alegando que este ha perdido su examen a pesar de prometer que han ido y entregado dicho examen. También es habitual, que algunos universitarios prometan haber asistido a una evento con reconocimiento de créditos, declarando que posteriormente no se les han convalidado dichos créditos. En ocasiones, el estudiante es culpable, ya sea por intento de fraude, mentira\dots Pero por desgracia también hay ocasiones en las que es inocente y en efecto ha asistido al examen o al evento y ha sido víctima de un problema de gestión de asistencias. \\

  No existe un único método para gestionar las asistencias a exámenes, charlas, talleres, prácticas\dots Ni existe un protocolo que todos los profesores u organizadores de eventos sigan al pie de la letra. De hecho, raramente se gestiona la asistencia a exámenes o prácticas, exceptuando alguna en la que se pide al alumno identificarse, pero sin llegar a registrarlo en ninguna parte. Tanto alumnos como profesores salen perdiendo, pues siempre queda en duda quien es culpable ante la teórica perdida de un examen o la teórica falta de asistencia a un evento. Perfectamente la solución a este problema puede ser pedir que los alumnos firmen una hoja, pero es tedioso, lleva tiempo, seguimos sin poder verificar la verdadera identidad del alumno y al igual que un examen, la hoja, puede desaparecer. Y es por eso mismo, que la mejor solución es usar la tecnología a nuestro favor. \\

``Estublock'' viene a resolver este problema, la aplicación permitirá registrar estas asistencias, escaneando un código QR y registrando la información en una red blockchain. Además, dará soporte para otros eventos como talleres, el congreso anual que se realiza en la escuela ``TryIT'', prácticas, charlas\dots. Con el tiempo, la aplicación irá creciendo y mejorando, trayendo mejoras poco a poco y con la capacidad de expandirse a otros campus de la Universidad Politécnica de Madrid y posteriormente crecer a otras universidades públicas para que todas puedan aprovechar el potencial de ``Estublock''. \\

Tanto por el bien del alumno, como del profesor, me motiva este proyecto pues ante todo me gusta que la vida sea justa, y me hace feliz saber que con esta aplicación evitaré los fraudes con los que se tiene que lidiar en el presente. Tanto alumnos irresponsables que no quieren aceptar la realidad, como docentes despistados a los que se les ha extraviado un examen. Además, hacer crecer este proyecto y ser capaces de validar mucho más que las asistencias a exámenes hace de este proyecto una oportunidad de mostrar todo lo aprendido, y poder verlo en funcionamiento en un futuro me motiva mucho a trabajar en ello. \\

También, parte de este TFG es el desarrollo de un ``kit de desarrollo software'' (SDK) el cual será de código abierto y disponible en repositorios públicos como GitHub o Maven para que otros desarrolladores en cualquier parte del planeta puedan utilizar este SDK. Esta contribución al mundo del software libre, a la ``Free Software Foundation'' es de gran interés personal pues creo que las innovaciones están para compartirlas y lograr que crezcan con la ayuda de toda la comunidad de desarrolladores interesados. \\

Además, tengo el gusto de trabajar en la Cátedra Inetum, en el Campus Blockchain, lo que me facilita la comunicación con profesionales en la materia del mundo del blockchain y la tecnología punta. Haciendo de esta, una gran oportunidad para aprender y hacer crecer mi circulo de relaciones. Así como la oportunidad de mejorar mi trabajo en equipo y aprender como es la vida en una empresa y con que problemas hay que lidiar a la hora de hacer crecer un proyecto.

\section{Objetivos}

En base a las necesidades que debe solventar la aplicación, se centran todos los objetivos principales que han sido desarrollados en este Trabajo de Fin de Grado.
\begin{itemize}
\item Desarrollo de una API que sirva como medio de comunicación entre la aplicación móvil que se va a desarrollar con él servidor de la base de datos que guarda información sobre los alumnos, detalles de las asignaturas\dots y con el servidor que ejecuta uno de los nodos de la red blockchain. 
\item Desarrollo de la aplicación móvil Android, la cual a parte de comunicarse con la API anteriormente mencionada, tendrá también que enviar a la red blockchain transacciones firmadas por el usuario, así como crear y guardar el wallet del usuario. 
\item Desarrollo de un SDK a partir de la aplicación anteriormente mencionada, este SDK contendrá la funcionalidad de comunicación con la red blockchain y de gestión del wallet del usuario. La aplicación móvil utilizará entonces este SDK una vez terminado. 
\item Desarrollo de un documentación para el correcto uso del SDK, y así facilitar a otros desarrolladores su uso.
\end{itemize}
Siendo de los puntos anteriormente mencionados, la Aplicación Android, el corazón del trabajo realizado.

\section{Estructura}
Para lograr cumplir con los objetivos propuestos, se ha divido la estructura del trabajo en las siguientes secciones. \\

El capítulo \emph{Estado del Arte} contiene una introducción a la tecnología Blockchain, haciendo especial hincapié en la red de \emph{Ethereum}\cite{webEthereum}. La razón de esta decisión es que la red que se ha utilizado para este proyecto es una red de Quorum\cite{webQuorum} la cual permite aprovechar el potencial de \emph{Ethereum} en aplicaciones blockchain. Además, se introducen al final del capítulo ejemplos de otras aplicaciones móviles existentes en el presente con funcionalidades diferentes pero que utilizan la red de \emph{Ethereum} para sus transacciones. Además se han añadido dos apartados de trabajo realizado durante el TFG. El primero es un apartado sobre la API que se ha desarrollado para la comunicación entre el móvil y la base de datos y blockchain. Este apartado se encuentra en el estado del arte puesto que se ha realizado en equipo. Y luego tenemos un apartado sobre conceptos básicos de Android que he ido aprendiendo durante el desarrollo del TFG. \\

El siguiente capítulo \emph{Desarrollo}, explica todo el diseño de la aplicación móvil tanto a nivel de interfaz de usuario como a nivel interno. Además, se hace hincapié en la comunicación con la red blockchain así como las librerías utilizadas en el proceso. Terminada la parte android, se pasa al desarrollo del SDK, su funcionamiento, el tratamiento del wallet. Terminando con la documentación del SDK. \\

Para cumplir con los objetivos de desarrollo sostenible, se ha añadido un capítulo \emph{Impacto Medioambiental} en el que se expone el impacto del uso de la aplicación desarrollada. \\

Por último, la \emph{Conclusión} y los \emph{Pasos a Futuro}, recogen los resultados del trabajo, mis conclusiones personales, y cuales son los pasos a seguir para hacer crecer al proyecto ``Estublock''.

