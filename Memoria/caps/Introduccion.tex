\chapter{Introducción}
\label{cap:Introduccion}

En esta sección se presenta el contexto del TFG, la motivación detrás de este trabajo, los objetivos a cumplir y una breve explicación de la estructura del documento.

\section{Contexto}

A lo largo de los dos últimos años, hemos tenido la suerte de estar trabajando en la \emph{Cátedra Inetum} de la Escuela técnica superior de ingenieros informáticos, en colaboración de la empresa Inetum. La cátedra Inetum se constituye para estrechar la colaboración entre la universidad y la empresa Inetum. Inetum es una compañía multinacional de servicios ágil que proporciona servicios y soluciones digitales y un grupo global que ayuda a compañías e instituciones a aprovechar al máximo la corriente digital. Durante estos dos años, hemos estado investigando sobre la tecnología blockchain y el potencial que puede aportar a las personas, en concreto enfocado al mundo universitario. \\

Dentro de esta cátedra, y con los conocimientos e investigaciones realizadas, se ha ideado desde cero un proyecto muy innovador. El proyecto ``Estublock'', tiene el objetivo de lanzar una red blockchain que todos los estudiantes puedan utilizar para desarrollar sus propias aplicaciones. Como primer producto se ha desarrollado la aplicación ``Estublock'', que surge ante la necesidad de un sistema fiable, robusto y rápido, de registro de asistencias a exámenes. A partir de esta necesidad, se ha expandido la funcionalidad de la aplicación para cubrir otros tipos de eventos como prácticas, talleres, laboratorios, etc. Con la meta, de hacer de Estublock el sistema estandar para validar la asistencia a exámenes, prácticas, talleres, etc de todas las Escuelas (y facultad) de la Universidad Politécnica de Madrid, y ofrecer la posibilidad de extenderlo al resto de universidades públicas y así asegurar una correcta educación.

\section{Motivación}

  Es habitual que, como parte de los procesos de evaluación, se consideren actividades que requieren de presencialidad (laboratorios, participación activa en el aula, etc.). La utilización de "listas de asistencia" presenta múltiples problemas, incluidos los de retraso en el comienzo de la actividad por el tiempo que requiere ese tipo de control, posible pérdida de los listados, sobre todo cuando se utiliza el papel como soporte, posibilidad de que se suplanten identidades, etc. El problema radica en que no hay una forma común, confiable y transparente de validar la asistencia a dichas pruebas. Esto es extrapolable a cualquier tipo de evento que requiera que se valide la asistencia (obtención de créditos de libre elección, etc.). \\

  No existe un único método para gestionar las asistencias a exámenes, charlas, talleres, prácticas, etc. Ni existe un protocolo que todos los profesores u organizadores de eventos sigan al pie de la letra. De hecho, raramente se gestiona la asistencia a exámenes o prácticas, exceptuando alguna en la que se pide al alumno identificarse, pero sin llegar a registrarlo en ninguna parte. Perfectamente la solución a este problema puede ser pedir que los alumnos firmen una hoja, pero es tedioso, lleva tiempo, seguimos sin poder verificar la verdadera identidad del alumno. Y al igual que un examen, la hoja puede desaparecer. Y es por eso mismo, que la mejor solución es usar la tecnología a nuestro favor. \\

``Estublock'' viene a resolver este problema. La aplicación permitirá registrar estas asistencias, escaneando un código QR y registrando la información en una red blockchain. Además, dará soporte para otros eventos como talleres, el congreso anual que se realiza en la escuela ``TryIT'', prácticas, charlas, etc. Con el tiempo, la aplicación irá creciendo y mejorando, trayendo mejoras poco a poco y con la capacidad de extenderlo a otros campus de la Universidad Politécnica de Madrid y posteriormente crecer a otras universidades públicas para que todas puedan aprovechar el potencial de ``Estublock''. \\

Tanto por el bien del alumno, como del profesor, este proyecto es muy motivador pues ante todo nos parece que la vida ha de ser justa, y es motivador saber que con esta aplicación se evitarán los fraudes con los que se tiene que lidiar en el presente. Tanto alumnos irresponsables que no quieren aceptar la realidad, como docentes despistados a los que se les ha extraviado un examen. Además, hacer crecer este proyecto y ser capaces de validar mucho más que las asistencias a exámenes hace de este proyecto una oportunidad de mostrar todo lo aprendido, y será un gran orgullo verlo en funcionamiento en un futuro. \\

También, parte de este TFG es el desarrollo de un ``kit de desarrollo software'' (SDK) el cual será de código abierto y disponible en repositorios públicos como GitHub o Maven para que otros desarrolladores en cualquier parte del planeta puedan utilizarlo. Esta contribución al mundo del software libre, es de gran interés personal pues las innovaciones en tecnología están para compartirlas y lograr que crezcan con la ayuda de toda la comunidad de desarrolladores interesados. \\

Además, disponemos de la oportunidad de trabajar en la Cátedra Inetum, en el Campus Blockchain, lo que facilita la comunicación con profesionales en la materia del mundo del blockchain y la tecnología punta. Haciendo de esta, una gran oportunidad para aprender y hacer crecer el circulo de relaciones profesionales. Así como la oportunidad de mejorar en el trabajo en equipo y aprender cómo es la vida en una empresa y con que problemas hay que lidiar a la hora de hacer crecer un proyecto en la vida real.

\section{Objetivos}

En base a las necesidades que debe solventar la aplicación, se centran todos los objetivos principales que han sido desarrollados en este Trabajo de Fin de Grado.
\begin{itemize}
\item Analizar profundamente la tecnología Blockchain, en concreto la red de Ethereum y sus smart contracts. Esto es importante pues la aplicación móvil desarrollada deberá ser capaz de comunicarse con una red \emph{Quorum} que aprovecha la red de Ethereum.
\item Desarrollo de una API que sirva como medio de comunicación entre la aplicación móvil que se va a desarrollar con él servidor de la base de datos que guarda información sobre los alumnos, detalles de las asignaturas, etc. Y con el servidor que ejecuta uno de los nodos de la red blockchain. 
\item Desarrollo de la aplicación móvil Android, la cual a parte de comunicarse con la API anteriormente mencionada, tendrá también que enviar a la red blockchain transacciones firmadas por el usuario, así como crear y guardar el wallet del usuario. 
\item Desarrollo de un SDK a partir de la aplicación anteriormente mencionada, este SDK contendrá la funcionalidad de comunicación con la red blockchain y de gestión del wallet del usuario. La aplicación móvil utilizará entonces este SDK una vez terminado. 
\item Desarrollo de un documentación para el correcto uso del SDK, y así facilitar a otros desarrolladores su uso, así como normas para el despliegue del mismo. 
\end{itemize}
Siendo de los puntos anteriormente mencionados, la Aplicación Android, el corazón del trabajo realizado.

\section{Estructura}
Para lograr cumplir con los objetivos propuestos, se ha divido la estructura del trabajo en las siguientes secciones. \\

El capítulo \emph{Estado del Arte}\ref{cap:EstadoArte} contiene una introducción a la tecnología Blockchain, haciendo especial hincapié en la red de \emph{Ethereum}\cite{webEthereum}. La razón de esta decisión es que la red que se ha utilizado para este proyecto es una red de Quorum\cite{webQuorum} la cual permite aprovechar el potencial de \emph{Ethereum} en aplicaciones blockchain. Además, se introducen al final del capítulo ejemplos de otras aplicaciones móviles existentes en el presente con funcionalidades diferentes pero que utilizan la red de \emph{Ethereum} para sus transacciones. Luego se han añadido dos apartados de trabajo realizado durante el TFG. El primero es un apartado sobre la API que se ha desarrollado para la comunicación entre el móvil y la base de datos y blockchain. Este apartado se encuentra en el estado del arte puesto que se ha realizado en equipo. Y, finalmente, se presenta un apartado sobre conceptos básicos de Android que se han ido aprendiendo a lo largo del desarrollo del TFG. \\

El siguiente capítulo \emph{Desarrollo}\ref{cap:Desarrollo}, explica todo el diseño de la aplicación móvil tanto a nivel de interfaz de usuario como a nivel interno. Además, se hace hincapié en la comunicación con la red blockchain así como las librerías utilizadas en el proceso. También se profundiza en el desarrollo del SDK, su funcionamiento, el tratamiento del wallet, y su uso en otras aplicaciones. Terminando con la documentación del SDK. Y por último un estudio sobre la seguridad de la aplicación y la seguridad del keystore. \\

Para considerar los objetivos de desarrollo sostenible, se ha añadido un capítulo \emph{Impacto Medioambiental}\ref{cap:ImpactoMedioAmbiente} en el que se expone el impacto del uso de la aplicación desarrollada. \\

Por último, la \emph{Conclusión}\ref{cap:Conclusiones} y los \emph{Pasos a Futuro}\ref{cap:Futuro}, recogen los resultados del trabajo, las conclusiones personales, y cuales son los pasos a seguir para hacer crecer al proyecto ``Estublock''.

