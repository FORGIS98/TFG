\pagestyle{plain}

\selectlanguage{spanish}
\cleardoublepage
\phantomsection
\addcontentsline{toc}{chapter}{Resumen}

\begin{abstract}
  % Resumen Introducción
  Durante los últimos dos años se ha tenido la oportunidad de trabajar en la Cátedra Inetum investigando y estudiando todo lo relacionado a la tecnología blockchain. Dentro de esta cátedra, surgió un proyecto llamado ``Estublock'' con el cual se pretende resolver un problema más común de lo deseado. Es habitual que todos los años algún alumno tenga algún problema con la asistencia o no a un examen. Esto se debe a que el profesor ha podido extraviar dicho examen, y el alumno no tiene manera de demostrar que realmente realizó el examen. La universidad se encuentra en la tesitura de creer al alumno que asegura haber realizado el examen o creer al profesor que asegura no tener dicho examen. El problema radica, en que no hay una forma común y fiable de validar la asistencia a exámenes (y otros eventos). \\

  % Resumen Estado del Arte
  Para solucionar este problema se quiere usar blockchain, esta es una tecnología permite registrar transacciones de manera distribuida. Como un libro contable distribuido. Transacciones monetarias, transacciones con información o datos como el nombre de una persona, la asistencia a un evento, un billete de avión, una entrada de cine\dots La red blockchain elegida para el trabajo es la red de Ethereum, gracias a que tiene la posibilidad de ejecutar Smart Contracts en ella. Los smart contracts son básicamente código que se ejecuta de manera automática ante una llamada a una de sus funciones. En el presente existen varios proyectos que aprovechan la red de Ethereum, como \emph{Guts, LifeID y Voatz}. Para desarrollar ``Estublock'' se ha utilizado Android, uno de los sistemas operativos más utilizados en el presente y para facilitar la comunicación con la base de datos y algunos procesos de la comunicación con la red blockchain, se ha desarrollado una API RESTful. \\

  % Resumen Desarrollo
  El desarrollo de la aplicación y del SDK han supuesto un gran reto. Se ha diseñado con Marvelapp un diseño de pantallas con la intención de que sirva como plantilla para después realizar en la aplicación móvil, con código XML, el diseño final de la aplicación. Junto a Marvelapp, se han diseñado unos diagramaas y unos casos de uso para visualizar el objetivo de la aplicación. Aunque los diseños de Marvelapp han sido de gran utilidad, muchas de las pantallas han sufrido cambios según se iban programando y según el proyecto iba creciendo. Con respecto al código, la aplicación se ha desarrollado utilizando Java y múltiples librerías para facilitar las llamadas a las APIs, así como llamadas a la red blockchain. Las llamadas a la red blockchain se han implementado en un SDK a parte, así, se podrá compartir en el repositorio Maven Central para que otros desarrolladores de aplicaciones móvil puedan utilizarlo. Al trabajar con Blockchain hay un concepto importante que se ha estudiado a lo largo del desarrollo, los wallets y los keystores. Piezas claves en las que los usuarios guardan a buen recaudo sus credenciales para firmar transacciones en la red blockchain. Además se han realizado pruebas a la aplicación utilizando la técnica \emph{White Box Testing}. \\

  % Resumen Conclusiones + Futuro
  La tecnología blockchain crece sin cesar, cada día surgen nuevos proyectos y también mejoras en el sistema, y aunque puede estar un poco verde, es sin duda alguna el futuro. A ``Estublock'' le queda un largo camino por recorrer, se ha logrado terminar una primera versión viable, la cual puede probarse en el presente para validar la asistencia de alumnos en alguna prueba académica o taller. Sin embargo, hay que seguir desarrollandola, añadiendo funcionalidades para el usuario, tanto visuales como técnicas. Añadir un poco más de documentación para que futuros desarrolladores puedan seguir con el trabajo hecho y preparar algunos tests para desarrollar con más seguridad y evitar fallos. Sin duda alguna, este proyecto nos ha enseñado mucho, y nos ha hecho ver lo mucho que queda por aprender y descubrir. \\

\textbf{Palabras Clave}: Blockchain, Smart Contract, Ethereum, Android, SDK, Web3j, Quorum, Java, Móvil, Librerías, Wallet, Keystore\dots
\end{abstract}

\selectlanguage{english}
\cleardoublepage
\phantomsection
\addcontentsline{toc}{chapter}{Abstract}

\begin{abstract}
  During the last two years, we have had the opportunity to work in the ``Cátedra Inetum'' researching and studying everything related to blockchain technology. Within this workplace, a project called ``Estublock'' was created to solve a problem that is more common than desired. It is common that every year some student has a problem with the attendance or not to an exam. This is because the professor may have misplaced the exam, and the student has no way of proving that he or she took the exam. The university is faced with the choice of believing the student who claims to have taken the exam or believing the professor who claims not to have taken the exam. The problem is that there is no common and reliable way to validate attendance at exams (and other events). \\

  To solve this problem we want to use blockchain, this is a technology that allows recording transactions in a distributed way. Like a distributed ledger. Monetary transactions, transactions with information or data such as a person's name, attendance at an event, a plane ticket, a movie ticket The blockchain network is chosen for the work is the Ethereum network, thanks to the fact that it can execute Smart Contracts on it. Smart contracts are code that executes automatically upon a call to one of its functions. At present, several projects take advantage of the Ethereum network, such as \emph{Guts, LifeID and Voatz}. To develop ``Estublock'' Android has been used, one of the most widely used operating systems at present and to facilitate communication with the database and some processes of communication with the blockchain network, a RESTful API has been developed. \\

  The development of the application and the SDK has been a great challenge. A screen design has been designed with Marvelapp to serve as a template to later make in the mobile application, with XML code, the final design of the application. Together with Marvelapp, some diagrams and use cases have been designed to visualize the objective of the application. Although the Marvelapp designs have been very useful, many of the screens have changed as they were programmed and as the project grew. Concerning the code, the application has been developed using Java and multiple libraries to facilitate API calls, as well as calls to the blockchain network. The blockchain network calls have been implemented in a separate SDK, so they can be shared in the Maven Central repository so that other mobile application developers can use it. When working with Blockchain there is an important concept that has been studied throughout the development, wallets, and key stores. Key pieces in which users keep their credentials safe to sign transactions in the blockchain network. In addition, the application has been tested using the \emph{White Box Testing} technique. \\

  Blockchain technology is growing steadily, new projects and system improvements are emerging every day, and although it may be a bit green, it is undoubtedly the future. ``Estublock'' has a long way to go, a first viable version has been completed, which can be tested at present to validate the attendance of students in an academic test or workshop. However, it needs to be further developed, adding both visual and technical functionalities for the user. Add a little more documentation so that future developers can continue with the work done and prepare some tests to develop with more security and avoid failures. Undoubtedly, this project has taught us a lot and has made us see how much there is to learn and discover. \\

  \textbf{Keywords}: Blockchain, Smart Contract, Ethereum, Android, SDK, Web3j, Quorum, Java, Mobile, Libraries, Wallet, Keystore\dots
\end{abstract}

\ifspanish
	\selectlanguage{spanish}
\else
	\selectlanguage{english}
\fi
