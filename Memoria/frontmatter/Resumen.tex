%--- Ajustes del documento.
\pagestyle{plain}	% Páginas sólo con numeración inferior al pie

% -------------------------
%
% RESUMEN:
% OJO: Si es preciso cambiar manualmente orden Resumen <-> Abstract
%
% -------------------------
%--- Resumen en español
\selectlanguage{spanish} % Selección de idioma del resumen.
\cleardoublepage % Se incluye para modificar el contador de página antes de añadir bookmark
\phantomsection  % Necesario con hyperref
\addcontentsline{toc}{chapter}{Resumen} % Añade al TOC.

\begin{abstract}
% EDITAR: Resumen (máx. 1 pág.)
\begin{center}
  Por suerte o por desgracia, es habitual que unos pocos universitarios presenten quejas todos los años contra algún profesor, alegando que este ha perdido su examen a pesar de prometer que han ido y entregado dicho examen. También es habitual, que algunos universitarios prometan haber asistido a una evento con reconocimiento de créditos, declarando que posteriormente no se les han convalidado dichos créditos. En ocasiones, el estudiante es culpable, ya sea por intento de fraude, mentira\dots Pero por desgracia también hay ocasiones en las que es inocente y en efecto ha asistido al examen o al evento y ha sido víctima de un problema de gestión de asistencias. \\
  \vspace{0.25 cm}
  No existe un único método para gestionar las asistencias a exámenes, charlas, talleres, prácticas\dots Ni existe un protocolo que todos los profesores u organizadores de eventos sigan al pie de la letra. De hecho, raramente se gestiona la asistencia a exámenes o prácticas, exceptuando alguna en la que se pide al alumno identificarse, pero sin llegar a registrarlo en ninguna parte. Tanto alumnos como profesores salen perdiendo, pues siempre queda en duda quien es culpable ante la teórica perdida de un examen o la teórica falta de asistencia a un evento. Perfectamente la solución a este problema puede ser pedir que los alumnos firmen una hoja, pero es tedioso, lleva tiempo, seguimos sin poder verificar la verdadera identidad del alumno y al igual que un examen, la hoja, puede desaparecer. Y es por eso mismo, que la mejor solución es usar la tecnología a nuestro favor. \\
  \vspace{0.25 cm}
   Blockchain es una tecnología de gran crecimiento en los últimos años, a grandes rasgos es como una base de datos distribuida, en la que cada nodo (ordenador) contiene una copia de toda la información de la red, y la actualiza con nuevos datos constantemente. Para facilitar la comunicación con las redes blockchain a la vez que solucionar el problema de registro de asistencia a exámenes y eventos, se ha desarrollado a lo largo de este trabajo una aplicación móvil para \textit{Android} junto con un \textit{kit de desarrollo software} (SDK). La aplicación móvil, permite a profesores crear eventos (exámenes, prácticas\dots) y a los alumnos asistir a estos eventos, validando su asistencia mediante un código QR que escanea y firmar el profesor automáticamente desde la aplicación. Toda la información queda registrada en una red blockchain levantada en la universidad, en concreto una red de \textit{quorum}. Y, como parte fundamental en el proyecto, esta el SDK que permite firmar y enviar información a la red blockchain desde el dispositivo móvil. Este SDK se encarga de crear las carteras virtuales a los usuarios, validar las contraseñas de las carteras, firmar transacciones y enviar datos a la red blockchain, todo esto con ayuda de la librería \textit{web3j}. Brindando así un sistema fiable, para el registro de asistencias. \\
  \vspace{0.25 cm}
   \textbf{Palabras Clave}: Blockchain, Smart Contract, Ethereum, Android, SDK, Web3j, Quorum, Java, Móvil, Librerías\dots
\end{center}
\end{abstract}
%---



%--- Resumen en inglés
% Abstract
\selectlanguage{english} % Selección de idioma del resumen.
\cleardoublepage
\phantomsection % Necesario con hyperref
\addcontentsline{toc}{chapter}{Abstract} % Añade al TOC.

\begin{abstract}
% EDITAR: Abstract (máx. 1 pág.)

\begin{center}
  Fortunately or unfortunately, it is common for a few university students to file complaints every year against a professor, claiming that they have lost their exam despite promising that they have attended and handed in the exam. It is also common for some university students to promise to have attended an event with credit recognition, stating that they have not had their credits granted. Sometimes the student is guilty, either by intent to defraud or lying. But unfortunately, there are also times when the student is innocent and has attended the exam or event, and has been the victim of an attendance management problem. \\
  \vspace{0.25 cm}
  There is no single method for managing attendance at exams, lectures, workshops, practices, etc. Nor is there a protocol that all professors or event organizers follow to the letter. Attendance at exams or practices is rarely managed, except for the odd one where the student is asked to identify him/herself but is not recorded anywhere. Both students and teachers lose out, as it is always in doubt who is to blame for the theoretical loss of an exam or the theoretical lack of attendance at an event. The solution to this problem may well be to ask students to sign a form, but it is tedious, it takes time, we are still unable to verify the true identity of the student, and just like an exam, the form can be misplaced. And that's why the best solution is to use technology to our advantage. \\
  \vspace{0.25 cm}
  Blockchain is a technology of great growth in recent years, roughly speaking it is like a distributed database, in which each node (computer) contains a copy of all the information in the network, and updates it with new data constantly. To facilitate communication with blockchain networks while solving the problem of recording attendance at exams and events, a mobile application for Android has been developed along with a software development kit (SDK). The mobile application allows teachers to create events (exams, practices) and students to attend these events, validating their attendance through a QR code that the teacher scans and signs automatically from the application. All the information is recorded in a blockchain network set up at the university, specifically a \textit{quorum} blockchain network. And, as a fundamental part of the project, there is the SDK that allows signing and sending information to the blockchain network from the mobile device. This SDK is responsible for creating virtual wallets for users, validating wallet passwords, signing transactions, and sending data to the blockchain network, all with the help of the \textit{web3j} library. Thus providing a reliable system for recording attendance. \\
  \vspace{0.25 cm}
  \textbf{Keywords}: Blockchain, Smart Contract, Ethereum, Android, SDK, Web3j, Quorum, Java, Mobile, Libraries\dots
\end{center}
\end{abstract}
%---

%--- Ajuste del idioma para el resto del documento.
\ifspanish
	\selectlanguage{spanish}% Emplea idioma español
\else
	\selectlanguage{english}% Emplea idioma inglés
\fi
