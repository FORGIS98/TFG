\pagestyle{plain}

\selectlanguage{spanish}
\cleardoublepage
\phantomsection
\addcontentsline{toc}{chapter}{Resumen}

\begin{abstract}
  % Resumen Introducción
  Durante los últimos dos años he tenido la oportunidad de trabajar en la Cátedra Inetum investigando y estudiando todo lo relacionado a la tecnología blockchain. Dentro de la iniciativa Campus Blockchain de esta Cátedra, surgió un proyecto llamado ``Estublock'' con el cual se pretende resolver un problema más común de lo esperable. Es habitual que, como parte de los procesos de evaluación, se consideren actividades que requieren de presencialidad (laboratorios, participación activa en el aula, etc.). La utilización de ``listas de asistencia'' presenta múltiples problemas, incluidos los de retraso en el comienzo de la actividad por el tiempo que requiere ese tipo de control, posible pérdida de los listados, sobre todo cuando se utiliza el papel como soporte, posibilidad de que se suplanten identidades, etc. El problema radica en que no hay una forma común, confiable y transparente de validar la asistencia a dichas pruebas. Esto es extrapolable a cualquier tipo de evento que requiera que se valide la asistencia (obtención de créditos de libre elección, etc). \\

  % Resumen Estado del Arte
  Para solucionar este problema se propone usar blockchain. Esta es una tecnología permite registrar transacciones de manera distribuida. Como un libro contable distribuido. Transacciones monetarias, transacciones con información o datos como el nombre de una persona, la asistencia a un evento, un billete de avión, una entrada de cine, etc. La red blockchain elegida para el trabajo es la red de Ethereum, debido a que ofrece la posibilidad de ejecutar Smart Contracts en ella. Los smart contracts son básicamente código que se ejecuta de manera automática ante una llamada a una de sus funciones. En el presente existen varios proyectos que aprovechan la red de Ethereum, como \emph{Guts, LifeID y Voatz}. \\

  % Resumen Desarrollo
  Para desarrollar ``Estublock'' se ha utilizado Android, uno de los sistemas operativos más utilizados en el presente y para facilitar la comunicación con la base de datos y algunos procesos de la comunicación con la red blockchain, se ha desarrollado una API RESTful. El desarrollo de la aplicación y del SDK han supuesto un gran reto, pues nunca antes había trabajado en este campo. Se ha diseñado con Marvelapp un diseño de pantallas con la intención de que sirva como plantilla para después realizar en la aplicación móvil, con código XML, el diseño final de la aplicación. Junto a Marvelapp, se han diseñado unos diagramaas y unos casos de uso para visualizar el objetivo de la aplicación. Aunque los diseños de Marvelapp han sido de gran utilidad, muchas de las pantallas han sufrido cambios según se iban programando y según el proyecto iba creciendo. Con respecto al código, la aplicación se ha desarrollado utilizando Java y múltiples librerías para facilitar las llamadas a las APIs, así como llamadas a la red blockchain. Las llamadas a la red blockchain se han implementado en un SDK a parte. Así, se podrá compartir en el repositorio Maven Central para que otros desarrolladores de aplicaciones móvil puedan utilizarlo. Al trabajar con Blockchain hay un concepto importante que se ha estudiado a lo largo del desarrollo, los wallets y los keystores. Piezas claves en las que los usuarios guardan a buen recaudo sus credenciales para firmar transacciones en la red blockchain. Además se han realizado pruebas a la aplicación utilizando la técnica \emph{White Box Testing}. \\

  % Resumen Conclusiones + Futuro
  La tecnología blockchain crece sin cesar, cada día surgen nuevos proyectos y también mejoras en el sistema, y aunque puede estar algo inmadura, es sin duda alguna el futuro. A ``Estublock'' le queda un largo camino por recorrer, se ha logrado terminar una primera versión viable, la cual puede probarse actualmente para validar la asistencia de alumnos en alguna prueba académica o taller. Sin embargo, hay que seguir desarrollándola, añadiendo funcionalidades para el usuario, tanto visuales como técnicas. Mejorar la documentación para que futuros desarrolladores puedan seguir con el trabajo hecho y preparar algunos tests para desarrollar con más seguridad y evitar fallos. Sin duda alguna, este proyecto me ha enseñado mucho, y me ha hecho ver lo mucho que queda por aprender y descubrir. \\

\textbf{Palabras Clave}: Blockchain, Smart Contract, Ethereum, Android, SDK, Web3j, Quorum, Java, Móvil, Librerías, Wallet, Keystore, etc.
\end{abstract}

\selectlanguage{english}
\cleardoublepage
\phantomsection
\addcontentsline{toc}{chapter}{Abstract}

\begin{abstract}
  During the last two years, I have had the opportunity to work in the \emph{Cátedra Inetum} researching and studying everything related to blockchain technology. Within the Campus Blockchain initiative of this \emph{Cátedra}, a project called ``Estublock'' emerged, with which is intended to solve a problem more common than expected. It is common that, as part of the evaluation processes, activities that require attendance (laboratories, active participation in the classroom, etc.) are considered. The use of ``attendance lists'' presents multiple problems, including delays in the start of the activity due to the time required for this type of control, possible loss of the lists, especially when a paper is used as a support, the possibility of identity theft, etc. The problem lies in the fact that there is no common, reliable, and transparent way of validating attendance at such tests. This can be extrapolated to any type of event that requires validation of attendance (obtaining elective credits, etc.). \\

  To solve this problem it is proposed to use blockchain. This is a technology allows recording transactions in a distributed manner. Like a distributed ledger. Monetary transactions, transactions with information or data such as a person's name, attendance to an event, a plane ticket, a movie ticket, etc. The blockchain network chosen for the work is the Ethereum network because it offers the possibility of executing Smart Contracts on it. Smart contracts are code that is executed automatically upon a call to one of its functions. At present, several projects leverage the Ethereum network, such as \emph{Guts, LifeID and Voatz}. \\

  To develop ``Estublock'' Android has been used, one of the most used operating systems at present and to facilitate the communication with the database and some processes of the communication with the blockchain network, a RESTful API has been developed. The development of the application and the SDK has been a great challenge, as I had never worked in this field before. A screen design has been designed with Marvelapp to serve as a template to later make in the mobile application, with XML code, the final design of the application. Together with Marvelapp, some diagrams and use cases have been designed to visualize the objective of the application. Although the Marvelapp designs have been very useful, many of the screens have changed as they were programmed and as the project grew. Concerning the code, the application has been developed using Java and multiple libraries to facilitate API calls, as well as calls to the blockchain network. The calls to the blockchain network have been implemented in a separate SDK. Thus, it can be shared in the Maven Central repository so that other mobile application developers can use it. When working with Blockchain there is an important concept that has been studied throughout the development, wallets, and keystores. Key pieces in which users keep their credentials safe to sign transactions in the blockchain network. In addition, the application has been tested using the \emph{White Box Testing} technique. \\

  Blockchain technology is growing steadily, new projects and also improvements in the system are emerging every day, and although it may be somewhat immature, it is undoubtedly the future. ``Estublock'' still has a long way to go, a first viable version has been completed, which can currently be tested to validate the attendance of students in an academic test or workshop. However, it is necessary to continue developing it, adding functionalities for the user, both visual and technical. Improve the documentation so that future developer can continue with the work done and prepare some tests to develop with more security and avoid failures. Undoubtedly, this project has taught me a lot and has made me see how much there is to learn and discover. \\

  \textbf{Keywords}: Blockchain, Smart Contract, Ethereum, Android, SDK, Web3j, Quorum, Java, Mobile, Libraries, Wallet, Keystore, etc.
\end{abstract}

\ifspanish
	\selectlanguage{spanish}
\else
	\selectlanguage{english}
\fi
