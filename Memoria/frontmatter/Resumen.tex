%--- Ajustes del documento.
\pagestyle{plain}	% Páginas sólo con numeración inferior al pie

% -------------------------
%
% RESUMEN:
% OJO: Si es preciso cambiar manualmente orden Resumen <-> Abstract
%
% -------------------------
%--- Resumen en español
\selectlanguage{spanish} % Selección de idioma del resumen.
\cleardoublepage % Se incluye para modificar el contador de página antes de añadir bookmark
\phantomsection  % Necesario con hyperref
\addcontentsline{toc}{chapter}{Resumen} % Añade al TOC.

\begin{abstract}
% EDITAR: Resumen (máx. 1 pág.)
\begin{center}
  % Resumen Introducción
  Durante los últimos dos años hemos tenido la oportunidad de trabajar en la Cátedra Inetum investigando y estudiando todo lo relacionado a la tecnología blockchain. Blockchain es una tecnología de gran crecimiento en los últimos años, a grandes rasgos es como una base de datos distribuida, en la que cada nodo (ordenador) contiene una copia de toda la información de la red, y la actualiza con nuevos datos constantemente. Dentro de esta cátedra, surgió un proyecto llamado ``Estublock'' con el cual se pretende resolver un problema más común de lo deseado, es habitual que todos los años algún alumno acuse a un profesor de perder su exámen. La universidad se encuentra ante la decisión de creer al alumno que promete haber asistido al exámen, o el profesor que promete no tener dicho exámen. El problema radica, en que no hay una forma común y fiable de validar la asistencia a exámenes (y otros eventos). Con la determinación de resolver este problema, se ha creado ``Estublock'' aprovechando la tecnología Blockchain. \\

  % Resumen Estado del Arte
  \vspace{0.25 cm}
  Como ya se ha mencionado, blockchain viene a ser una base de datos distribuida que básicamente guarda transacciones. Transacciones monetarias, transacciones con información o datos como el nombre de una persona, un billete de avión, un entrada de cine\dots Todas las redes blockchain, al cabo de un tiempo o al cabo de un conjunto de transacciones, pasa a validar estas mismas generando un nuevo bloque. Para validar la información y poder añadir el nuevo bloque, según el algorítmo de consenso que utilice, los nodos de la red se ponen a trabajar. Por ejemplo, en el algorítmo Proof of Work todos los ordenadores tratarán de resolver una función hash determinada, y quien antes lo consiga, podrá añadir el nuevo bloque a la red. La red blockchain elegida para el trabajo es la red de Ethereum, gracias a que tiene la posibilidad de ejecutar Smart Contracts en ella. Los smart contracts son básicamente código que se ejecuta de manera automática ante una llamada a una de sus funciones. En el presente existen varios proyectos que aprovechan la red de Ethereum, como \emph{Guts, LifeID y Voatz}. Para desarrollar ``Estublock'' se ha utilizado Android, uno de los sistemas operativos más utilizados en el presente y para facilitar la comunicación con la base de datos y algunos procesos de la comunicación con la red blockchain, se ha desarrollado una API RESTful. \\

  % Resumen Desarrollo
  \vspace{0.25 cm}
  TODO MY FRIEND \\

  % Resumen Conclusiones + Futuro
  \vspace{0.25 cm}
  La tecnología blockchain crece sin cesar, cada día surgen nuevos proyectos y también mejoras en el sistema, y aunque puede estar un poco verde, es sin duda alguna el futuro. A ``Estublock'' le queda un largo camino por recorrer, se ha logrado terminar una primera versión viable, la cual puede probarse en el presente para validar la asistencia de alumnos en alguna prueba académica o taller. Sin embargo, hay que seguir desarrollandola, añadiendo funcionalidades para el usuario, tanto visuales como técnicas. Añadir un poco más de documentación para que futuros desarrolladores puedan seguir con el trabajo hecho y preparar algunos tests para desarrollar con más seguridad y evitar fallos. Sin duda alguna, este proyecto nos ha enseñado mucho, y nos damos cuenta de lo mucho que nos queda por aprender. \\
  \vspace{0.25 cm}
   \textbf{Palabras Clave}: Blockchain, Smart Contract, Ethereum, Android, SDK, Web3j, Quorum, Java, Móvil, Librerías, Wallet\dots
\end{center}
\end{abstract}
%---



%--- Resumen en inglés
% Abstract
\selectlanguage{english} % Selección de idioma del resumen.
\cleardoublepage
\phantomsection % Necesario con hyperref
\addcontentsline{toc}{chapter}{Abstract} % Añade al TOC.

\begin{abstract}
% EDITAR: Abstract (máx. 1 pág.)

\begin{center}
  During the last two years, we have had the opportunity to work in the ``Cátedra Inetum'' researching and studying everything related to blockchain technology. Blockchain is a technology of great growth in recent years, roughly speaking it is like a distributed database, in which each node (computer) contains a copy of all the information on the network, and updates it with new data constantly. Within this workplace, a project called ``Estublock'' was created to solve a problem that is more common than desired, it is common that every year a student accuses a professor of losing his exam. The university is faced with the decision of believing the student who promises to have attended the exam, or the professor who promises not to have the exam. The problem is that there is no common and reliable way to validate attendance at exams (and other events). With the determination to solve this problem, ``Estublock'' has been created leveraging Blockchain technology. \\

  \vspace{0.25 cm}
  As already mentioned, blockchain is a distributed database that stores transactions. Monetary transactions, transactions with information or data such as a person's name, a plane ticket, a movie ticket, etc. All blockchain networks, after some time or after a set of transactions, validate these transactions by generating a new block. To validate the information and to be able to add the new block, depending on the consensus algorithm used, the nodes of the network start working. For example, in the Proof of Work algorithm, all the computers will try to solve a given hash function, and whoever succeeds first will be able to add the new block to the network. The blockchain network chosen for the work is the Ethereum network, thanks to the fact that it can execute Smart Contracts on it. Smart contracts are code that is executed automatically upon a call to one of its functions. At present, several projects take advantage of the Ethereum network, such as emph{Guts, LifeID and Voatz}. To develop ``Estublock'' Android has been used, one of the most widely used operating systems at present and to facilitate communication with the database and some processes of communication with the blockchain network, a RESTful API has been developed. \\

  \vspace{0.25 cm}
  TODO MY FRIEND \\

  \vspace{0.25 cm}
  Blockchain technology is growing steadily, new projects and system improvements are emerging every day, and although it may be a bit green, it is undoubtedly the future. Estublock'' has a long way to go, a first viable version has been completed, which can be tested in the present to validate the attendance of students in an academic test or seminar. However, it needs to be further developed, adding both visual and technical functionalities for the user. Add a little more documentation so that future developers can continue with the work done and prepare some tests to develop with more security and avoid failures. Undoubtedly, this project has taught us a lot, and we realize how much we still have to learn. \\
  \vspace{0.25 cm}
  \textbf{Keywords}: Blockchain, Smart Contract, Ethereum, Android, SDK, Web3j, Quorum, Java, Mobile, Libraries, Wallet\dots
\end{center}
\end{abstract}
%---

%--- Ajuste del idioma para el resto del documento.
\ifspanish
	\selectlanguage{spanish}% Emplea idioma español
\else
	\selectlanguage{english}% Emplea idioma inglés
\fi
