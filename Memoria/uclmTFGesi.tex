% CABECERA
% BEGIN_FOLD
%%%%%%%%%%%%%%
% Fichero: uclmTFGesi.tex
% Autor: Jesús Salido Tercero (https://www.esi.uclm.es/www/jsalido)
% Fecha (creación): Febrero 2010 
% Rev. : Febrero 2021
% Descripción: Plantilla para memoria de TFG 
% (Escuela Sup. de Informática, UCLM). Creada para el curso 
% “LaTeX esencial para preparación de TFG, Tesis y otros documentos 
% académicos” (Esc. Sup. Informática-UCLM)
%
%### Compilación 
%
% Esta plantilla ha sido preparada para compilarse con `pdflatex`, `biblatex` 
% (bibliografía con `biber`) y `makeindex` (sólo si se incluye índice 
% temático).
%
% Para su compilación se aconseja utilizar `latexmk` (requiere para su 
% ejecución de un intérprete perl:
%
%> \$> latexmk -gg -pdflatex -bibtex-cond1 -silent -auxdir=build -outdir=build uclmTFGesi.tex
%
% Para la automatización del trabajo con esta plantilla es recomendable el 
% empleo de IDE dedicados como [TeXstudio](https://www.texstudio.org/).
%
% Una versión revisada de esta plantilla está disponible en overleaf.
% Puede crearse un proyecto propio para escribir un TFG directamente en 
% overleaf, o bien descargarla como un archivo .zip para su utilización en 
% modo local.
%
% Si deseas acceder a la versión de desarrollo puedes encontrarla en GitHub:
%	https://github.com/JesusSalido/TFG_ESI_UCLM
%%%%%%%%%%%%%%
% END_FOLD


% -------------------------
%
% PREÁMBULO
% BEGIN_FOLD
% -------------------------
\documentclass[ 		% Clase del documento
	11pt,				% Tamaño de letra
	a4paper,			% Tamaño de papel
	twoside,			% Impresión a doble cara
	openright,			% La apertura de cap. a la dcha.
%	draft       		% Versión borrador (sin figuras)
	final       		% Versión final
]{book}

%--- Codificación de entrada (mejora respecto a inputenc)
\usepackage[utf8]{inputenx} 
\usepackage[
    english,     % Se emplea porque el resumen siempre en inglés
    spanish,     % Se emplea para resumen en español
    es-tabla,    % Si idioma pral. español
    es-noindentfirst
]{babel} % Internacionalización


%--- Geometría de las páginas del documento
\usepackage[			% Márgenes del documento
	top=2.5cm,			% Margen superior
	bottom=2.5cm,		% Margen inferior
	inner=3.5cm,		% Margen al interior
	outer=2cm			% Margen al exterior
]{geometry}

%--- Tipografía
\usepackage{amsmath,amssymb,amsfonts} % Para ecs.

%--- Si no se emplea ningún paquete de tipografía se empleará Computern Modern.
%--- Tipografía 
%--- (Opción: Latin Modern)
%\usepackage{lmodern} % Latin Modern. Empleada cuando se desea una tipografía genuina de LaTeX sucesora de Computer Modern.
%--- (Opción: Libertine)
\usepackage[tt=false]{libertine} % Libertine con Old-Style Figures [osf]
\usepackage[libertine]{newtxmath} % Times
%--- (Opción: Palatino)
%\usepackage{newpxtext} % Palatino: La opción osf proporciona números en old style.
%\usepackage{newpxmath}	% Palatino
%--- (Opción: Fourier)
%\usepackage{fourier} % Utopía
%---

%--- (Opción excepcional)
% Si es preciso cambio de tipo de familia de tipografía por defecto a Sans-Serif
% Aunque es una opción extraña es la preferida en algunos centros docentes (ADE-UCLM).
% Con esta elección es más conveniente una tipografía tipo Helvética/Arial (no Libertine)
%\usepackage{helvet}
%\renewcommand{\familydefault}{\sfdefault}
%---

\usepackage{textcomp,marvosym,pifont} % OPT.: Generación de símbolos 
%especiales
\usepackage{ccicons} % OPT.: Iconos de licencia Creative Commons
\usepackage[T1]{fontenc}% Codificación de salida    

%--- Paquete con personalización local para el TFG (ESI-UCLM)
% EDITAR: Si es preciso que la memoria emplee "english" como idioma 
% principal, prefijos de género y ubicación en número de página al pie.
% Con la opción "english" el resto de opciones son irrelevantes.
% Los prefijos de género permiten un tratamiento adecuado en portadas
% automáticas.
%
% Opciones del paquete: (por defecto) Memoria en español y prefijos masculinos.
%	spanish: idioma pral. español (valor por defecto).
%   english: idioma pral. inglés.
%	autora,tutora,cotutora: indica el género de intervinientes (masculino por defecto) 	
%	pageonfooter: Número de página en el pie y centrado como ADE-UCLM (por defecto en cabeceras)
% -------------------------
% COMANDOS OPCIONALES PROPORCIONADOS:
% OP-Opcional
% RE-Recomendado
%
% Comandos de generación automática de portadas
% \portadaTFG, \portadaTFG*{ficheroPDF}: Página de portada
% \portillaTFG, \portadillaTFG*{ficheroPDF}: Página de contraportada
% \tribunalTFG, \tribunalTFG*{ficheroPDF}: Página de tribunal
% \dedicado{texto}: Dedicatoria
% \creditos{texto}{fichero gráfico de licencia}
%
% Comandos para definir variables con los datos del documento.
%
% OP-\nodivide[penalty]			: Penaliza la división de palabras. Máx. 
%								: (n=10000) sin arg.
% OP-\nowidowandorphan[penalty] : Penaliza las viudas y huérfanas. (sólo si necesario)
% OP-\nodividenotes[penalty]	: Penaliza la división de notas al pié entre págs.	(sólo si necesario)	
% OP-\savepagecnt				: Salva en un contador interno el nº de pág. actual
% OP-\contpagination			: Recupera el valor de pág. previamente salvado en el cont. interno.
% OP-\tecla{texto}				: Genera un borde de tecla en torno al 
%texto		
% -------------------------
%\usepackage{uclmTFGesi}
\usepackage[tutora,cotutora]{uclmTFGesi}
%\usepackage[pageonfooter]{uclmTFGesi}



%--- Bibliografía: Biblatex con biber.
% Permite cambiar los estilos de citación y ordenación de la bibliografía
\usepackage[%
	backend=biber,
	% Estilos: numeric, numeric-comp, alphabetic, authoryear, authoryear-comp
	% Otros: apa, chicago, ieee, mla-new, iso-numeric, iso-authoryear
	% Estilos tradicionales BibTeX: trad-plain, trad-unsrt, trad-alpha y trad-abbrv
%	bibstyle=apa, % Estilo APA
	bibstyle=ieee,
	% Citación: numeric, numeric-comp, numeric-verb, 
	%           authoryear, authoryear-comp,...
	%           otros aplicados con estilo gral.: ieee,apa,aml,chem-acs,iso-numeric,iso-authoryear
	citestyle=numeric-comp,
	sortcites, % Ordenación de citas múltiples cuando son numéricas
	maxbibnames=3, % Máximo número listado de autores en la bibliografía
	minbibnames=1, % Mínimo número de autores cuando se abrevia la lista de autores
	% Descomentar las opciones siguientes para bibliografía multilingüe
	autolang=other, % Requerido para opción multilingüe
	language=auto,   % Requerido para opción multilingüe
	sorting=nyt%
					% Para cambiar criterio de ordenación de las referencias.
 					% =nty (name-title-year), nyt (name-year-title), nyvt (name-year-volume-title), 
					% =anyt (alphabetic-name-year-title), anyvt (alphabetic-name-year-volume-title), 
					% =ynt (year-name-title), ydnt (yeardescendent-name-title), 
					% =none (por orden de citación, como en ETSII-UCLM).			
]{biblatex}

% Línea añadida para eliminar el idioma de la fuente bibliográfica.
\AtEveryBibitem{\clearfield{note} \clearlist{language}}

% OPT: No comentar si es necesario suprimir el sangrado de párrafos.
%\setlength{\parindent}{0pt} % Elimina sangrado de párrafos

\addbibresource{biblioTFG.bib} 	% Fichero de bibliografía.

\DeclareGraphicsExtensions{.pdf,.png,.jpg} % Precedencia de extensiones
\graphicspath{{./figs/}}% Path de búsqueda de ficheros gráficos

\usepackage{makeidx} % Indice temático

% END_FOLD
% EDITAR: Descomentar si se desea índice temático
%\makeindex           % OPT.: Procesamiento de índice temático
% -------------------------
% -------------------------
% -------------------------



% DATOS del documento
% EDITAR: Datos del documento. 
% Los elementos opcionales pueden dejarse, eliminarse o comentarse. No se deben dejar vacíos porque provoca errores.
% BEGIN_FOLD
% -------------------------
% -------------------------
% -------------------------
% DATOS DEL DOCUMENTO 
% Definición de variables empleadas en el documento por lo que no son
% traducidos. Cuando algún campo puede tener varias líneas aparecen dos
% campos señalados como <campo>Primera y <campo>Segunda. Si no se desea 
% emplear los campos opcionales (OPT.) estos deben comentarse.
% -------------------------
\tituloPrimera{Diseño e Implementación de un SDK Android para Facilitar la Interacción de Aplicaciones Móvil con una Blockchain} % 1ª Línea
% \tituloSegunda{Curso de {\LaTeX{}} esencial} % OPT.: Para títulos largos.
\titulo{SDK Android con interacción con Blockchain} % Título corto (mostrado en pág. de créditos)
\autor{Jorge Sol Gonzalez}
\email{jorge.sol.gonzalez@alumnos.upm.es}					
\tutor{Francisco javier Soriano Camino} % Sólo nombre, el prefijo añadido automát.
% \cotutor{<co-tutor(a) (nombre apellidos)>}	% OPT.: Cotutor(a)
\instEdu{UNIVERSIDAD POLITÉCNICA DE MADRID}

% Fichero con escudo de la institución
% Logo de la ESI que prefieras (la normativa no especifica obligatoriedad).
%\escudo{esi} 					% Logo ESI (gris uniforme)							
%\escudo{esi_black} 			% Logo ESI (negro)						
%\escudo{esi_color}				% Logo ESI (dos tintas)						
\escudo{IngInformatica_color}	% Nucleo de ferrita	(color)
%\escudo{IngInformatica_bw} 	% Nucleo de ferrita	(en escala de grises)					

\centroEdu{ESCUELA TÉCNICA SUPERIOR DE INGENIEROS INFORMÁTICOS}
\deptoEduPrimera{Departamento de Lenguajes y Sistemas} % 1ª Línea(EN: Department of ...)
% \deptoEduSegunda{<Segunda línea Depto. Director>} % OPT.: Para nombres largos.
\titulacion{GRADO EN INGENIERÍA INFORMÁTICA} % (EN: BACHELOR IN COMPUTING ENGINEERING)
% \especialidad{<Tecnología Específica>} % OPT.: Especialidad - Intensificación
% (EN: Specialization in ...)
\tipoDoc{TRABAJO FIN DE GRADO} % (EN: BACHELOR DISSERTATION)

% Si las fechas se desean en inglés hay que ponerla explícita.
\fechaDef{junio, 2021} 			% Fecha de defensa
\mesDef{junio}        			% Mes de defensa
\yearDef{2021}        			% Año de defensa
\lugarDef{Madrid}			% Lugar de defensa


% --- Metadatos (propiedades) para el documento PDF
\hypersetup{% OPT.
	pdftitle={Diseño e Implementación de un SDK Android para Facilitar la Interacción de Aplicaciones Móvil con una Blockchain}, % Título
	pdfauthor={Jorge Sol Gonzalez}, % Autor
	pdfsubject={TFG},  % Tema
	pdftoolbar=true, % Muestra la toolbar de Acrobat
	pdfmenubar=true	 % Muestra la menubar de Acrobat
}
% -------------------------
% -------------------------
% -------------------------
% END_FOLD



% -------------------------
% -------------------------
% -------------------------
% -------------------------
%
% CUERPO del documento
%
% -------------------------
\begin{document}
%--- PORTADAS + FRONTMATTER
\frontmatter
% Cambia la numeración de páginas a números romanos y las secciones no están 
%numeradas aunque si aparecen en el índice de contenidos.
\pagestyle{empty}  % Páginas sin cabecera ni pies

\includecomment{portadas} % Cambiar por exclude para evitar compilar portadas
%\excludecomment{portadas} % Cambiar por exclude para evitar compilar portadas
\begin{portadas}
% -------------------------
% -------------------------
% -------------------------
%
% PORTADAS: 
%
% Los comandos \portadaTFG y \portadilla generan dos portadas con LaTeX 
% teniendo en cuenta los datos sobre el documento aportados en el preámbulo.
% Si se desea incluir una portada generada externamente en PDF se emplea la
% versión del comando con estrella indicando el fichero.
%
% -------------------------





%---
% \portadaTFG		% Portada pral.
%\portadaTFG*{./figs/portadaETSII} % Portada generada externamente en PDF
% También en versión con estrella para indicar fichero PDF
% Comentar si no se desea incluir
%---






%---
\portadillaTFG	% Portada interior (con tutor(a) y co-tutor(a) si existe).
% También en versión con estrella para indicar fichero PDF
% Comentar si no se desea incluir
%---





%---
% OPT.: CRÉDITOS (aunque no es obligatorio es recomendable).
% -------------------------
%
% CRÉDITOS
%
% -------------------------
% EDITAR: El autor puede elegir el tipo de licencia que desee para distribuir su TFG que puede variar con respecto a la de este documento en el que si permitimos obra derivada para que no surjan dudas sobre la reutilización del material.
% Este comando permite una gran flexibilidad y la ventaja de no depender de paquetes externos.
% Esta es una página reservada para señalar información relativa a los derechos de autor y la licencia de distribución y uso del documento. Esta página debería ser aprovechada también para informar de cualquier tipo de cesión de los derechos anteriormente citados. El autor del TFG debe tener presente que el incumplimiento de la legislación vigente en materia de protección de la propiedad intelectual es de su exclusiva responsabilidad independientemente de la cesión de derechos que este haya convenido para su obra ya que no son objeto de cesión aquellos derechos de los que no se es poseedor.

\creditos{Este documento se distribuye con licencia CC BY-NC-SA 4.0. El texto completo de la licencia puede obtenerse en \url{https://creativecommons.org/licenses/by-nc-sa/4.0/}.

La copia y distribución de esta obra está permitida en todo el mundo, sin regalías y por cualquier medio, siempre que esta nota sea preservada. Se concede permiso para copiar y distribuir traducciones de este libro desde el español original a otro idioma, siempre que la traducción sea aprobada por el autor del libro y tanto el aviso de copyright como esta nota de permiso, sean preservados en todas las copias.

% NOTA: Deja esta nota de atribución al uso de la plantilla LaTeX.
Este texto ha sido preparado con la plantilla \LaTeX{} de TFG para la UCLM publicada por \href{https://www.esi.uclm.es/www/jsalido}{Jesús Salido} en GitHub\footnote{\url{https://github.com/JesusSalido/TFG_ESI_UCLM}} y Overleaf \footnote{\url{https://www.overleaf.com/latex/templates/plantilla-de-tfg-escuela-superior-de-informatica-uclm/phjgscmfqtsw}} como parte del curso \href{http://visilab.etsii.uclm.es/?page_id=1468}{\emph{<<\LaTeX{} esencial para preparación de TFG, Tesis y otros documentos académicos>>}} impartido en la Escuela Superior de Informática de la Universidad de Castilla-La Mancha.}{by-nc-sa}


% NOTA: Para citar este documento puede emplear el registro siguiente:
%@www{salidoTFG,
%  author       = {Jesús Salido},
%  title        = {Plantilla guía de TFG para la ESI-UCLM},
%  year         = {2019},
%  editor       = {GitHub},
%  organization = {Universidad de Castilla-La Mancha},
%  url          = {https://github.com/JesusSalido/TFG_ESI_UCLM},
%}


%---



%---




%---
% \tribunalTFG % Página para calificaciones del tribunal
% También en versión con estrella para indicar fichero PDF
%---





%---
% OPT.: DEDICATORIA (1 pág. máximo) comentar si no se desea incluir.
% Aunque opcional, no se debería perder la oportunidad de poder 
% dedicar el trabajo a alguien MUY especial.
% EDITAR: Dedicatoria.
\dedicado{A mis estudiantes \\ % A alguien muy especial
Por contribuir a hacer de cada día un reto ilusionante} % Como mucho dos líneas (no confundir con los agradecimientos).
%---
%
% FIN PORTADAS: 
% |
% |
% -> ----------------------
\end{portadas}

% Puedes comentar los apartados que no deseas compilar
\pagestyle{plain}

\selectlanguage{spanish}
\cleardoublepage
\phantomsection
\addcontentsline{toc}{chapter}{Resumen}

\begin{abstract}
  % Resumen Introducción y Estado del Arte
  Durante los últimos dos años se ha tenido la oportunidad de trabajar en la Cátedra Inetum investigando y estudiando todo lo relacionado a la tecnología blockchain. Dentro de esta cátedra, surgió un proyecto llamado ``Estublock'' con el cual se pretende resolver un problema más común de lo deseado. Es habitual que todos los años algún alumno acuse a un profesor de perder su exámen. La universidad se encuentra ante la decisión de creer al alumno que promete haber asistido al exámen, o el profesor que promete no tener dicho exámen. El problema radica, en que no hay una forma común y fiable de validar la asistencia a exámenes (y otros eventos). Para solucionar este problema se quiere usar blockchain, esta tecnología viene a ser como una base de datos distribuida que guarda transacciones. Transacciones monetarias, transacciones con información o datos como el nombre de una persona, un billete de avión, un entrada de cine\dots La red blockchain elegida para el trabajo es la red de Ethereum, gracias a que tiene la posibilidad de ejecutar Smart Contracts en ella. Los smart contracts son básicamente código que se ejecuta de manera automática ante una llamada a una de sus funciones. En el presente existen varios proyectos que aprovechan la red de Ethereum, como \emph{Guts, LifeID y Voatz}. Para desarrollar ``Estublock'' se ha utilizado Android, uno de los sistemas operativos más utilizados en el presente y para facilitar la comunicación con la base de datos y algunos procesos de la comunicación con la red blockchain, se ha desarrollado una API RESTful. \\

  % Resumen Desarrollo
  El desarrollo de la aplicación y del SDK han supuesto un gran reto. Se trata de un campo que nunca antes se había trabajado. Se ha diseñado con Marvelapp un diseño de pantallas con la intención de que sirva como plantilla para despues realizar en la aplicación móvil, con código XML, el diseño final de la aplicación. Aunque ha sido muy útil tener a mano los diseños de Marvelapp, muchas de las pantallas han sufrido cambios según se iban programando y según el proyecto iba creciendo. Con respecto al código, la aplicación se ha desarrollado utilizando Java y múltiples librerías para facilitar las llamadas a las APIs, así como llamadas a la red blockchain. Las llamadas a la red blockchain se han implementado en un SDK a parte, así, se podrá compartir en el repositorio Maven Central para que otros desarrolladores de aplicaciones móvil puedan utilizarlo. También se han estudiado diferentes tipos de wallet, eligiendo las \emph{Software Wallet} como la ideal y se han estudiado métodos de recuperación de wallets como las \emph{Seed Phrases}. \\

  % Resumen Conclusiones + Futuro
  La tecnología blockchain crece sin cesar, cada día surgen nuevos proyectos y también mejoras en el sistema, y aunque puede estar un poco verde, es sin duda alguna el futuro. A ``Estublock'' le queda un largo camino por recorrer, se ha logrado terminar una primera versión viable, la cual puede probarse en el presente para validar la asistencia de alumnos en alguna prueba académica o taller. Sin embargo, hay que seguir desarrollandola, añadiendo funcionalidades para el usuario, tanto visuales como técnicas. Añadir un poco más de documentación para que futuros desarrolladores puedan seguir con el trabajo hecho y preparar algunos tests para desarrollar con más seguridad y evitar fallos. Sin duda alguna, este proyecto nos ha enseñado mucho, y nos ha hecho ver lo mucho que queda por aprender y descubrir. \\

\textbf{Palabras Clave}: Blockchain, Smart Contract, Ethereum, Android, SDK, Web3j, Quorum, Java, Móvil, Librerías, Wallet\dots
\end{abstract}

\selectlanguage{english}
\cleardoublepage
\phantomsection
\addcontentsline{toc}{chapter}{Abstract}

\begin{abstract}

  During the last two years, we have had the opportunity to work in the ``Cátedra Inetum'' researching and studying everything related to blockchain technology. Within this workplace, a project called ``Estublock'' was created to solve a problem that is more common than desired. It is common every year for a student to accuse a professor of missing an exam. The university is faced with the decision of believing the student who promises to have attended the exam, or the professor who promises not to have the exam. The problem is that there is no common and reliable way to validate attendance at exams (and other events). To solve this problem we want to use blockchain, this technology is like a distributed database that stores transactions. Monetary transactions, transactions with information or data such as a person's name, a plane ticket, a movie ticket\dots The blockchain network chosen for the work is the Ethereum network, thanks to the fact that it has the possibility of executing Smart Contracts on it. Smart contracts are code that executes automatically upon a call to one of its functions. At present, several projects take advantage of the Ethereum network, such as \emph{Guts, LifeID and Voatz}. To develop ``Estublock'' Android has been used, one of the most used operating systems at present and to facilitate communication with the database and some processes of communication with the blockchain network, a RESTful API has been developed. \\

  The development of the application and the SDK has been a great challenge. This is a field that had never been worked before. A screen design has been designed with Marvelapp to serve as a template to later create the final design of the application in the mobile application, with XML code. Although it has been very useful to have the Marvelapp designs at hand, many of the screens have changed as they were being programmed and as the project grew. Regarding the code, the application has been developed using Java and multiple libraries to facilitate API calls, as well as calls to the blockchain network. The calls to the blockchain network have been implemented in a separate SDK, so it can be shared in the Maven Central repository for other mobile application developers to use it. Different wallet types have also been studied, choosing the \emph{Software Wallet} as the ideal one, and wallet recovery methods such as \emph{Seed Phrases} have been studied. \\

  Blockchain technology is growing steadily, new projects and system improvements are emerging every day, and although it may be a bit green, it is undoubtedly the future. Estublock'' has a long way to go, a first viable version has been completed, which can be tested in the present to validate the attendance of students in an academic test or seminar. However, it needs to be further developed, adding both visual and technical functionalities for the user. Add a little more documentation so that future developers can continue with the work done and prepare some tests to develop with more security and avoid failures. Without a doubt, this project has taught us a lot and has made us realize how much there is still to learn and discover. \\

  \textbf{Keywords}: Blockchain, Smart Contract, Ethereum, Android, SDK, Web3j, Quorum, Java, Mobile, Libraries, Wallet\dots
\end{abstract}

\ifspanish
	\selectlanguage{spanish}
\else
	\selectlanguage{english}
\fi
 % Abstract
\ifspanish
	\selectlanguage{spanish}
\else
	\selectlanguage{english}
\fi

% -------------------------
%
% AGRADECIMIENTOS (recomendable máx. 1 pág.)
%
% -------------------------
\cleardoublepage
\phantomsection % Necesario con hyperref

\chapter*{Agradecimientos} % Opción con * para que no aparezca en TOC ni numerada
\addcontentsline{toc}{chapter}{Agradecimientos} % Añade al TOC.

Mil gracias a todos mis amigos, especialmente a los de la universidad, con los que he convivido en las buenas y en las malas durante la carrera. Gracias a Ferrero, Anto, Younes, Paula, Carlos, mis dos Diegos, Gaspar y Alex, Kalili y Borja, porque aunque no se lo crean, todos me han apoyado de una u otra manera a lo largo de estos años y de esta aventura que llamamos vida. \\

Gracias a los profesores, especialmente a Angel Herranz, que me ha enseñado lo importante que es tener curiosidad por lo desconocido. Gracias también a Victor Ramperez, porque por muy ocupado que este, siempre esta ahí cuando le necesitas. \\

Gracias a Roberto García, Antonio González y María Salgado, por enseñarme, ayudarme y hacerme crecer como profesional. Gracias a Paula Pousa, por el increíble equipo que hemos hecho juntos, por su paciencia, dedicación, y alegría. \\

Y más importante aún, gracias a mis padres y a mi hermana, por quererme, aceptarme y apoyarme. Especialmente, gracias a mi madre, por darme la oportunidad todos los días de crecer como persona. 

\makeatletter		
\begin{flushright}
	\vspace{1,5cm}
	\textit{\@autor}\\
	\@lugarDef, \@yearDef
\end{flushright}
\makeatother
 % Agradecimientos etc.
% -------------------------
%
% -NOTACIÓN: Lista de símbolos con significado especial.
%
% -------------------------
\cleardoublepage
\phantomsection % Necesario con hyperref

% El método mostrado en este fichero es un modo rápido de incluir nomeclatura y listade acrónimos. En trabajos donde se precise un trabajo más depurado e intensivo puede recurrirse a los paquetes:
%   - nomencl
%   - acronym

\chapter*{Notación y acrónimos} % Opción con * para que no aparezca en TOC ni numerada
\addcontentsline{toc}{chapter}{Notación y acrónimos} % Añade al TOC.

\section*{Notacion}
Ejemplo de lista con notación (o nomenclatura) empleada en la memoria del TFG.\footnote{Se incluye únicamente con propósito de ilustración, ya que el documento no emplea la notación aquí mostrada.}

\begin{tabular}{r r p{0.8\linewidth}}
$A, B, C, D$	& : & Variables lógicas \\
$f, g, h$		& :	& Funciones lógicas \\
$\cdot$			& : & Producto lógico (AND), a menudo se omitirá como en $A 
B$ en lugar de $A \cdot B$\\
$+$				& : & Suma aritmética o lógica (OR) dependiendo del 
contexto\\
$\oplus$		& : & OR exclusivo (XOR)\\
$\overline{A}$ o ${A}'$	& : & Operador NOT o negación
\end{tabular}

\section*{Lista de acrónimos}
Ejemplo de lista con los acrónimos empleados en el texto.

\begin{tabular}{r r p{0.8\linewidth}}
CASE& : &Computer-Aided Software Engineering \\
CTAN& : &Comprenhensive \TeX{} Archive network \\
IDE& : &Integrated Development Environment \\
ECTS& : &European Credit Transfer and Accumulation System \\
OOD& : &Object-Oriented Design \\
PhD& : &Philosophiae Doctor \\
RAD& : &Rapid Application Development \\
SDLC& : &Software Development Life Cycle \\
SSADM& : &Structured Systems Analysis \& Design Method \\
TFE& : &Trabajo Fin de Estudios \\
TFG& : &Trabajo Fin de Grado \\
TFM& : &Trabajo Fin de Máster \\
UML& : &Unified Modeling Language
\end{tabular} % Notación empleada.
% -------------------------
%
% ÍNDICES: 
% EDITAR: Si alguno de los índices no existe, su inclusión se puede comentar.
%
% -------------------------
\setindexnames % Ajusta nombres (sólo en español).
\pagestyle{fancy} % Estilo de página ajustado por fancyhdr

%--- Índice general
\cleardoublepage
\phantomsection % Necesario con hyperref
\pdfbookmark[0]{Índice general}{idx_toc}% idx_toc.0 % Bookmark en PDF
\tableofcontents  % Índice general
% Todos los listados se han incluido en el índice gral. de contenidos. De 
%modo automático también quedan añadidos a los bookmarks del PDF. Si se 
%desean 
%eliminiar del TOC se pueden comentar el comando \addcontensline.
%---

%--- Índice de figuras
\cleardoublepage
\phantomsection % Necesario con hyperref
\addcontentsline{toc}{chapter}{\listfigurename} % Añade la lista de figuras al TOC (también a bookmarks en PDF)
%\pdfbookmark[0]{\listfigurename}{idx_lof}% idx_lof.0 % Bookmark en PDF
\listoffigures    % Índice de figuras (opcional)
%---

% %--- Índice de tablas
% \cleardoublepage
% \phantomsection % Necesario con hyperref
% \addcontentsline{toc}{chapter}{\listtablename} % Añade la lista de tablas al TOC (también a bookmarks en PDF)
% %\pdfbookmark[0]{\listtablename}{idx_lot}% idx_lot.0 % Bookmark en PDF
% \listoftables % Índice de tablas (opcional)
% %---

% %--- Índice de listados
% % Comentar todo el bloque para no incluir.
% \cleardoublepage
% \phantomsection % Necesario con hyperref
% \addcontentsline{toc}{chapter}{\lstlistlistingname} % Añade la lista de listados al TOC (también a bookmarks en PDF)
% %\pdfbookmark[0]{\lstlistlistingname}{idx_lol}% idx_lol.0 % Bookmark en PDF
% \lstlistoflistings % Índice de listados creados con listings (opcional)
% %---

%% %--- Índice de algoritmos
%% % Comentar todo el bloque para no incluir.
%% \cleardoublepage
%% \phantomsection % Necesario con hyperref
%% \addcontentsline{toc}{chapter}{\listalgorithmcfname} % Añade la lista de algoritmos al TOC (también a bookmarks en PDF)
%% %\pdfbookmark[0]{\listalgorithmcfname}{idx_loa}% idx_loa.0 % Bookmark en PDF
%% \listofalgorithms % Índice de algoritmos creados con algortihm2e
%% %---

 % Índice de contenido, figuras, tablas, listados, etc.
%--- (FIN FRONTMATTER)



% Ajustes previos a Capítulos (MAINMATTER)
% BEGIN_FOLD
%--- MAINMATTER
% Capítulos del documento
% Salva en un contador interno el nº de páginas actual
% Debe ir antes de \mainmatter (antes de que se reinicie el cnt page)
\savepagecnt
\mainmatter
% Justo antes del primer capítulo del libro. Activa la numeración con números arábigos y reinicia el contador de páginas.

% Ajusta valor de cabeceras y pies a comienzo de capítulo
\ifpageonfooter
\else
	\cleanhdfirst
\fi

% Reajuste del número de página consecutivo para no reiniciar paginación en Cáp. 1
%\contpagination % Comentado para reiniciar paginación (pag. 1)
% END_FOLD

% -------------------------
%
% CAPÍTULOS: Un fichero por capítulo.
%
% -------------------------

% NOTA: En la ESI-UCLM es obligado interlineado de una línea, pero en algunas circunstancias se puede necesitar alterar dicho interlineado.
% OPT: Si deseas modificar el interlineado de una línea (por defecto)
% puedes emplear los comandos señalados (proporcionados por paquete `setspace')
%\onehalfspacing % Ajusta interlineado a 1.5 líneas
%\doublespacing  % Ajusta interlineado a 2 líneas  

\chapter{Introducción}
\label{cap:Introduccion}

En esta sección se presenta el contexto del TFG, la motivación detrás de este trabajo, los objetivos a cumplir y una breve explicación de la estructura del documento.

\section{Contexto}

A lo largo de estos dos últimos años, hemos tenido la suerte de estar trabajando en la \emph{Cátedra Inetum} de la escuela técnica superior de ingenieros informáticos con la colaboración de la empresa Inetum. Inetum es una compañía de servicios ágil que proporciona servicios y soluciones digitales y un grupo global que ayuda a compañías e instituciones a aprovechar al máximo la corriente digital. Durante estos dos años, hemos estado investigando sobre la tecnología blockchain y el potencial que puede aportar a las personas, en concreto enfocado al mundo universitario. \\

Dentro de esta cátedra, y con los conocimientos e investigaciones realizadas, se ha ideado desde cero un proyecto muy innovador. ``Estublock'', surge ante la necesidad de un sistema fiable, robusto y rápido, de registro de asistencias a exámenes. A partir de esta necesidad, se ha expandido la funcionalidad de la aplicación para cubrir otros tipos de eventos como prácticas, talleres, laboratorios\dots Y la meta, es hacer de Estublock él sistema por defecto para validar la asistencia a exámenes, prácticas, talleres\dots de todas las facultades de la Universidad Politécnica de Madrid, y expandir al resto de universidades públicas. 

\section{Motivación}

  Por suerte o por desgracia, es habitual que unos pocos universitarios presenten quejas todos los años contra algún profesor, alegando que este ha perdido su examen a pesar de prometer que han ido y entregado dicho examen. También es habitual, que algunos universitarios prometan haber asistido a una evento con reconocimiento de créditos, declarando que posteriormente no se les han convalidado dichos créditos. En ocasiones, el estudiante es culpable, ya sea por intento de fraude, mentira\dots Pero por desgracia también hay ocasiones en las que es inocente y en efecto ha asistido al examen o al evento y ha sido víctima de un problema de gestión de asistencias. \\

  No existe un único método para gestionar las asistencias a exámenes, charlas, talleres, prácticas\dots Ni existe un protocolo que todos los profesores u organizadores de eventos sigan al pie de la letra. De hecho, raramente se gestiona la asistencia a exámenes o prácticas, exceptuando alguna en la que se pide al alumno identificarse, pero sin llegar a registrarlo en ninguna parte. Tanto alumnos como profesores salen perdiendo, pues siempre queda en duda quien es culpable ante la teórica perdida de un examen o la teórica falta de asistencia a un evento. Perfectamente la solución a este problema puede ser pedir que los alumnos firmen una hoja, pero es tedioso, lleva tiempo, seguimos sin poder verificar la verdadera identidad del alumno y al igual que un examen, la hoja, puede desaparecer. Y es por eso mismo, que la mejor solución es usar la tecnología a nuestro favor. \\

``Estublock'' viene a resolver este problema, la aplicación permitirá registrar estas asistencias, escaneando un código QR y registrando la información en una red blockchain. Además, dará soporte para otros eventos como talleres, el congreso anual que se realiza en la escuela ``TryIT'', prácticas, charlas\dots. Con el tiempo, la aplicación irá creciendo y mejorando, trayendo mejoras poco a poco y con la capacidad de expandirse a otros campus de la Universidad Politécnica de Madrid y posteriormente crecer a otras universidades públicas para que todas puedan aprovechar el potencial de ``Estublock''. \\

Tanto por el bien del alumno, como del profesor, este proyecto es muy motivador pues ante todo nos parece que la vida a de ser justa, y es motivador saber que con esta aplicación se evitarán los fraudes con los que se tiene que lidiar en el presente. Tanto alumnos irresponsables que no quieren aceptar la realidad, como docentes despistados a los que se les ha extraviado un examen. Además, hacer crecer este proyecto y ser capaces de validar mucho más que las asistencias a exámenes hace de este proyecto una oportunidad de mostrar todo lo aprendido, y será un gran orgullo verlo en funcionamiento en un futuro. \\

También, parte de este TFG es el desarrollo de un ``kit de desarrollo software'' (SDK) el cual será de código abierto y disponible en repositorios públicos como GitHub o Maven para que otros desarrolladores en cualquier parte del planeta puedan utilizar este SDK. Esta contribución al mundo del software libre, a la ``Free Software Foundation'' es de gran interés personal pues las innovaciones en tecnología están para compartirlas y lograr que crezcan con la ayuda de toda la comunidad de desarrolladores interesados. \\

Además, disponemos de la oportunidad de trabajar en la Cátedra Inetum, en el Campus Blockchain, lo que facilita la comunicación con profesionales en la materia del mundo del blockchain y la tecnología punta. Haciendo de esta, una gran oportunidad para aprender y hacer crecer el circulo de relaciones profesionales. Así como la oportunidad de mejorar en el trabajo en equipo y aprender como es la vida en una empresa y con que problemas hay que lidiar a la hora de hacer crecer un proyecto en la vida real.

\section{Objetivos}

En base a las necesidades que debe solventar la aplicación, se centran todos los objetivos principales que han sido desarrollados en este Trabajo de Fin de Grado.
\begin{itemize}
\item Desarrollo de una API que sirva como medio de comunicación entre la aplicación móvil que se va a desarrollar con él servidor de la base de datos que guarda información sobre los alumnos, detalles de las asignaturas\dots y con el servidor que ejecuta uno de los nodos de la red blockchain. 
\item Desarrollo de la aplicación móvil Android, la cual a parte de comunicarse con la API anteriormente mencionada, tendrá también que enviar a la red blockchain transacciones firmadas por el usuario, así como crear y guardar el wallet del usuario. 
\item Desarrollo de un SDK a partir de la aplicación anteriormente mencionada, este SDK contendrá la funcionalidad de comunicación con la red blockchain y de gestión del wallet del usuario. La aplicación móvil utilizará entonces este SDK una vez terminado. 
\item Desarrollo de un documentación para el correcto uso del SDK, y así facilitar a otros desarrolladores su uso.
\end{itemize}
Siendo de los puntos anteriormente mencionados, la Aplicación Android, el corazón del trabajo realizado.

\section{Estructura}
Para lograr cumplir con los objetivos propuestos, se ha divido la estructura del trabajo en las siguientes secciones. \\

El capítulo \emph{Estado del Arte}\ref{cap:EstadoArte} contiene una introducción a la tecnología Blockchain, haciendo especial hincapié en la red de \emph{Ethereum}\cite{webEthereum}. La razón de esta decisión es que la red que se ha utilizado para este proyecto es una red de Quorum\cite{webQuorum} la cual permite aprovechar el potencial de \emph{Ethereum} en aplicaciones blockchain. Además, se introducen al final del capítulo ejemplos de otras aplicaciones móviles existentes en el presente con funcionalidades diferentes pero que utilizan la red de \emph{Ethereum} para sus transacciones. Luego se han añadido dos apartados de trabajo realizado durante el TFG. El primero es un apartado sobre la API que se ha desarrollado para la comunicación entre el móvil y la base de datos y blockchain. Este apartado se encuentra en el estado del arte puesto que se ha realizado en equipo. Y luego se presenta un apartado sobre conceptos básicos de Android que se han ido aprendiendo a lo largo del desarrollo del TFG. \\

El siguiente capítulo \emph{Desarrollo}\ref{cap:Desarrollo}, explica todo el diseño de la aplicación móvil tanto a nivel de interfaz de usuario como a nivel interno. Además, se hace hincapié en la comunicación con la red blockchain así como las librerías utilizadas en el proceso. También se profundiza en el desarrollo del SDK, su funcionamiento, el tratamiento del wallet, y su uso en otras aplicaciones. Terminando con la documentación del SDK. \\

Para cumplir con los objetivos de desarrollo sostenible, se ha añadido un capítulo \emph{Impacto Medioambiental}\ref{cap:ImpactoMedioAmbiente} en el que se expone el impacto del uso de la aplicación desarrollada. \\

Por último, la \emph{Conclusión}\ref{cap:Conclusiones} y los \emph{Pasos a Futuro}\ref{cap:Futuro}, recogen los resultados del trabajo, las conclusiones personales, y cuales son los pasos a seguir para hacer crecer al proyecto ``Estublock''.


\chapter{Objetivo}
\label{cap:Objetivo}

Introduce y motiva la problemática (i.e.\emph{\ ¿cuál es el problema que se plantea y por qué es interesante su resolución?})

Debe concretar y exponer detalladamente el problema a resolver, el entorno de 
trabajo, la situación y qué se pretende obtener. También puede contemplar las 
limitaciones y condicionantes a considerar para la resolución del problema 
(lenguaje de construcción, equipo físico, equipo lógico de base o de apoyo, 
etc.). Si se considera necesario, esta sección puede titularse 
\emph{Objetivos del TFG e hipótesis de trabajo}. En este caso, se añadirán 
las hipótesis de trabajo que el alumno pretende demostrar con su TFG.\index{TFG}

Una de las tareas más complicadas al proponer un TFG es plantear su \textsf{Objetivo}. La dificultad deriva de la falta de consenso respecto de lo que se entiende por \emph{objetivo} en un trabajo de esta naturaleza. En primer lugar se debe distinguir entre dos tipos de objetivo:

\begin{enumerate}
	\item La \emph{finalidad específica} del TFG que se plantea para resolver una problemática concreta aplicando los métodos y herramientas adquiridos durante la formación académica. Por ejemplo, \emph{<<Desarrollo de una aplicación software para gestionar reservas hoteleras \emph{on-line}>>}.
	
	\item El \emph{propósito académico} que la realización de un TFG tiene en la formación de un graduado. Por ejemplo, la \emph{adquisición de competencias específicas de la especialización} cursada.
\end{enumerate}

En el ámbito de la memoria del TFG se tiene que definir el primer tipo de objetivo, mientras que el segundo tipo es el que se añade al elaborar la propuesta de un TFG presentada ante un comité para su aprobación. \emph{Este segundo tipo de objetivo no debe incluirse en la memoria y en todo caso solo debe hacerse en la sección de conclusiones finales.}\footnote{En lagunas titulaciones es obligatorio que la memoria explique las competencias específicas alcanzadas con la realización del trabajo.}

Un objetivo bien planteado debe estar determinado en términos del \emph{<<producto final>>} esperado que resuelve un problema específico. Es por tanto un sustantivo que debería ser \emph{concreto} y \emph{medible}. El \textsf{Objetivo} planteado puede pertenecer una de las categorías que se indica a continuación:
\begin{itemize}
	\item \emph{Diseño y desarrollo de <<artefactos>>}\index{artefacto} 
	(habitual en las ingenierías),
	\item \emph{Estudio} que ofrece información novedosa sobre un tema (usual en las ramas de ciencias y humanidades), y
	\item \emph{Validación\index{validación} de una 
	hipótesis}\index{hipótesis} de partida (propio de los trabajos 
	científicos y menos habitual en el caso de los TFG).
\end{itemize}

Estas categorías no son excluyentes, de modo que es posible plantear un trabajo cuyo objetivo sea el diseño y desarrollo de un <<artefacto>> y este implique un estudio previo o la validación de alguna hipótesis para guiar el proceso. En este caso y cuando el objetivo sea lo suficientemente amplio puede ser conveniente su descomposición en elementos más simples hablando de \emph{subobjetivos}. Por ejemplo, un programa informático puede descomponerse en módulos o requerir un estudio previo para plantear un nuevo algoritmo que será preciso validar. La descomposición de un objetivo principal en subobjetivos\index{subobjetivo} 
u objetivos secundarios debería ser natural (no forzada), bien justificada y 
sólo pertinente en los trabajos de gran amplitud.

Junto con la definición del objetivo del trabajo se puede especificar los \emph{requisitos} que debe satisfacer la solución aportada. Estos requisitos especifican \emph{características} que debe poseer la solución y \emph{restricciones} que acotan su alcance. En el caso de un trabajo cuyo objetivo es el desarrollo de un <<artefacto>> los requisitos pueden ser \emph{funcionales} y \emph{no funcionales}.

Al redactar el objetivo de un TFG se debe evitar confundir los medios con el 
fin. Así es habitual encontrarse con objetivos definidos en términos de las 
\emph{acciones} (verbos) o \emph{tareas}\index{tarea} que será preciso 
realizar para llegar al verdadero objetivo. Sin embargo, a la hora de 
planificar el desarrollo del trabajo si es apropiado descomponer todo el 
trabajo en \emph{hitos} y estos en \emph{tareas} para facilitar dicha 
\emph{planificación}.

La categoría del objetivo planteado justifica modificaciones en la organización genérica de la memoria del trabajo. Así en el caso de estudios y validación de hipótesis el apartado de resultados y conclusiones debería incluir los resultados de experimentación y los comentarios de cómo dichos resultados validan o refutan la hipótesis planteada.


\chapter{Metodología}
\label{cap:Metodologia}

En este capítulo se debe detallar las metodologías\index{metodología} 
empleadas para planificación y desarrollo del trabajo, así como explicar de 
modo claro y conciso cómo se han aplicado dichas metodologías.

A continuación se incluye una guía rápida que puede ser de gran utilidad en la elaboración de este capítulo.

\section{Guía rápida de las metodologías de desarrollo de software}

\subsection{Proceso de desarrollo de software}

El \textbf{proceso de desarrollo de software} se denomina también 
\textbf{ciclo de vida del desarrollo del software} (\emph{SDLC, Software 
Development Life-Cycle})\index{SDLC}\index{Life-Cycle} y cubre las siguientes 
actividades:

\begin{enumerate}
\item \textbf{Obtención y análisis de requisitos}\index{requisitos} 
(\emph{requirements analysis}). Es la definición de la funcionalidad del 
software a desarrollar. Suele requerir entrevistas entre los ing. de software 
y el cliente para obtener el `qué' y `cómo'. Permite obtener una 
\emph{especificación funcional} del software.

\item \textbf{Diseño} (\emph{SW design}).\index{diseño} Consiste en la 
definición de la arquitectura, los componentes, las interfaces y otras 
características del sistema o sus componentes.

\item \textbf{Implementación} (\emph{SW construction and coding}). Es el
  proceso de codificación del software en un lenguaje de programación.
  Constituye la fase en que tiene lugar el desarrollo de software.

\item \textbf{Pruebas} (\emph{testing and verification}).\index{testing} 
Verificación del correcto funcionamiento del software para detectar fallos lo 
antes posible. Persigue la obtención de software de calidad. Consisten en 
pruebas de \emph{caja negra} y \emph{caja blanca}. Las primeras comprueban 
que la funcionalidad es la esperada y para ello se verifica que ante un 
conjunto amplio de entradas, la salida es correcta. Con las segundas se 
comprueba la robustez del código sometiéndolo a pruebas cuya finalidad es 
provocar fallos de software. Esta fase también incorpora la \emph{pruebas de 
integración} en las que se verifica la interoperabilidad del sistema con 
otros existentes.

\item \textbf{Documentación} (\emph{documentation}). Persigue facilitar la mejora continua del software y su mantenimiento.

\item \textbf{Despliegue} (\emph{deployment}).\index{despliegue} Consiste en 
la instalación del software en un entorno de producción y puesta en marcha 
para explotación. En ocasiones implica una fase de \emph{entrenamiento} de 
los usuarios del software.

\item \textbf{Mantenimiento} (\emph{maintenance}).\index{mantenimiento} Su 
propósito es la resolución de problemas, mejora y adaptación del software en 
explotación.
\end{enumerate}




\subsection{Metodologías de desarrollo software}
\emph{Las metodologías son el modo en que las fases del proceso software se organizan e interaccionan para conseguir que dicho proceso sea reproducible y predecible para aumentar la productividad y la calidad del software.}

Una metodología es una colección de:

\begin{itemize}
\item \textbf{Procedimientos} (indican cómo hacer cada tarea y en qué momento),
\item \textbf{Herramientas} (ayudas para la realización de cada tarea), y
\item \textbf{Ayudas documentales}.
\end{itemize}

Cada metodología es apropiada para un tipo de proyecto dependiendo de sus 
características técnicas, organizativas y del equipo de trabajo. En los 
entornos empresariales es obligado, a veces, el uso de una metodología 
concreta (p.~ej. para participar en concursos públicos). El estándar 
internacional ISO/IEC 12270\index{ISO/IEC 12270} describe el método para 
seleccionar, implementar y monitorear el ciclo de vida del software.

Mientras que unas intentan sistematizar y formalizar las tareas de diseño, otras aplican técnicas de gestión de proyectos para dicha tarea. Las metodologías de desarrollo se pueden agrupar dentro de varios enfoques según se señala a continuación.

\begin{enumerate}
\item \textbf{Metodología de Análisis y Diseño de Sistemas Estructurados} 
(\emph{SSADM, Structured Systems Analysis and Design 
Methodology}).\index{SSADM} Es uno de los paradigmas más antiguos. En esta 
metodología se emplea un modelo de desarrollo en cascada 
(\emph{waterfall}).\index{modelo!waterfall@\emph{waterfall}} Las fases de 
desarrollo tienen lugar de modo secuencial. Una fase comienza cuando termina 
la anterior. Es un método clásico poco flexible y adaptable a cambios en los 
requisitos. Hace especial hincapié en la planificación derivada de una 
exhaustiva definición y análisis de los requisitos. Son metodologías que no 
lidian bien con la flexibilidad requerida en los proyectos de desarrollo 
software. Derivan de los procesos en  ingeniería tradicionales y están 
enfocadas a la reducción del riesgo. Emplea tres técnicas clave:

\begin{itemize}
\item Modelado lógico de datos (\emph{Logical Data 
Modelling}),\index{modelado}
\item Modelado de flujo de datos (\emph{Data Flow Modelling}), y
\item Modelado de Entidades y Eventos (\emph{Entity Event
  Modelling}).
\end{itemize} 

\item \textbf{Metodología de Diseño Orientado a Objetos} (\emph{OOD,  
Object-Oriented Design}).\index{OOD} Está muy ligado a la \index{OOP}OOP (Programación 
Orientada a Objetos) en que se persigue la reutilización. A diferencia del 
anterior, en este paradigma los datos y los procesos se combinan en una única 
entidad denominada \emph{objetos} (o clases). Esta orientación pretende que 
los sistemas sean más modulares para mejorar la eficiencia, calidad del 
análisis y el diseño. Emplea extensivamente el Lenguaje Unificado de Modelado 
(UML)\index{UML} para especificar, visualizar, construir y documentar los 
artefactos de los sistemas software y  también el modelo de negocio. UML 
proporciona una serie diagramas de básicos para modelar un sistema: 

\begin{itemize}
\item Diagrama de Clase (\emph{Class Diagram}). Muestra los objetos del sistema y sus relaciones. 
\item Diagrama de Caso de Uso (\emph{Use Case Diagram}). Plasma la
  funcionalidad del sistema y quién interacciona con él.
\item Diagrama de secuencia (\emph{Sequence Diagram}). Muestra los eventos que se
  producen en el sistema y como este reacciona ante ellos. 
\item Modelo de Datos (\emph{Data Model}).
\end{itemize} 
                               
\item \textbf{Desarrollo Rápido de Aplicaciones} (\emph{RAD, Rapid 
Application Developmnent}).\index{RAD} Su filosofía es sacrificar calidad a 
cambio de poner en producción el sistema rápidamente con la funcionalidad 
esencial. Los procesos de especificación, diseño e implementación son 
simultáneos. No se realiza una especificación detallada y se reduce la 
documentación de diseño. El sistema se diseña en una serie de pasos, los 
usuarios evalúan cada etapa en la que proponen cambios y nuevas mejoras. Las 
interfaces de usuario se desarrollan habitualmente mediante sistemas 
interactivos de desarrollo. En vez de seguir un modelo de desarrollo en 
cascada sigue un modelo en espiral (Boehm).\index{modelo!espiral} La clave de 
este modelo es el desarrollo continuo que ayuda a minimizar los riesgos. Los 
desarrolladores deben definir las características de mayor prioridad. Este 
tipo de desarrollo se basa en la creación de prototipos y realimentación 
obtenida de los clientes para definir e implementar más características hasta 
alcanzar un sistema aceptable para despliegue.

\item \textbf{Metodologías Ágiles}. \emph{"[...] envuelven un enfoque para la 
toma de decisiones en los proyectos de software, que se refiere a métodos de 
ingeniería del software basados en el desarrollo iterativo e incremental, 
donde los requisitos y soluciones evolucionan con el tiempo según la 
necesidad del proyecto. Así el trabajo es realizado mediante la colaboración 
de equipos auto-organizados y multidisciplinarios, inmersos en un proceso 
compartido de toma de decisiones a corto plazo. Cada iteración del ciclo de 
vida incluye:  planificación, análisis de requisitos, diseño, codificación, 
pruebas y  documentación. Teniendo gran importancia el concepto de 
"Finalizado" (Done), ya que el objetivo de cada iteración no es agregar toda 
la funcionalidad para justificar el lanzamiento del producto al mercado, sino 
incrementar el valor por medio de "software que funciona" (sin errores). Los 
métodos ágiles enfatizan las comunicaciones cara a cara en vez de la 
documentación. [...]"}\footnote{Fuente: Wikipedia}\index{Wikipedia}
\end{enumerate}

\subsection{Proceso de testing}\index{testing}

\begin{enumerate}
\item \emph{Pruebas modulares} (pruebas unitarias). Su propósito es hacer pruebas sobre un módulo tan pronto como sea posible. Las \emph{pruebas unitarias} que comprueban el correcto funcionamiento de una unidad de código. Dicha unidad elemental de código consistiría en cada función o procedimiento, en el caso de programación estructurada y cada clase, para la programación orientada a objetos. Las características de una prueba unitaria de calidad son: \emph{automatizable} (sin intervención manual), \emph{completa},  \emph{reutilizable}, \emph{independiente} y \emph{profesional}.

\item \emph{Pruebas de integración}. Pruebas de varios módulos en conjunto para comprobar su interoperabilidad.

\item \emph{Pruebas de caja negra}.

\item \emph{Beta testing}.

\item \emph{Pruebas de sistema y aceptación}.

\item \emph{Training}.
\end{enumerate}






\subsection{Herramientas CASE (\emph{Computer Aided Software Engineering})}

Las herramientas CASE\index{CASE} están destinadas a facilitar una o varias 
de las tareas implicadas en el ciclo de vida del desarrollo de software. Se 
pueden dividir en la siguientes categorías:

\begin{enumerate}
\item Modelado y análisis de negocio.
\item Desarrollo. Facilitan las fases de diseño y construcción.
\item Verificación y validación.
\item Gestión de configuraciones.
\item Métricas y medidas.
\item Gestión de proyecto. Gestión de planes, asignación de tareas, planificación, etc.
\end{enumerate}




\subsubsection{IDE (Integrated Development Environment)}\index{IDE}
\begin{multicols}{2}
\begin{itemize}
\item \href{https://notepad-plus-plus.org/}{Notepad++}
\item \href{https://code.visualstudio.com/}{Visual Studio Code}
\item \href{https://atom.io/}{Atom}
\item \href{https://www.gnu.org/s/emacs/}{GNU Emacs}
\item \href{https://netbeans.org/}{NetBeans}
\item \href{https://eclipse.org/}{Eclipse}
\item \href{https://www.qt.io/ide/}{Qt Creator}
\item \href{http://www.jedit.org/}{jEdit}
\item \href{https://www.jetbrains.com/idea/}{ItelliJ IDEA}
\end{itemize}
\end{multicols}



\subsubsection{Depuración}\index{depuración}
\begin{itemize}
\item \href{https://www.gnu.org/s/gdb/}{GNU Debugger}
\end{itemize}


\subsubsection{Testing}\index{testing}
\begin{multicols}{2}
\begin{itemize}
\item \href{http://junit.org}{JUnit}. Entorno de pruebas para Java.
\item \href{http://cunit.sourceforge.net/}{CUnit}. Entorno de pruebas para C.
\item \href{https://wiki.python.org/moin/PyUnit}{PyUnit}. Entorno de pruebas para Python.
\end{itemize}
\end{multicols}

\subsubsection{Repositorios y control de versiones}\index{control de 
versiones}
\begin{multicols}{2}
\begin{itemize}
\item \href{https://git-scm.com/}{Git}
\item \href{https://www.mercurial-scm.org/}{Mercurial}
\item \href{https://github.com/}{Github}
\item \href{https://bitbucket.org/}{Bitbucket}
\item \href{https://www.sourcetreeapp.com/}{SourceTree}
\end{itemize}
\end{multicols}


\subsubsection{Documentación}
\begin{multicols}{2}
\begin{itemize}
\item \href{https://www.latex-project.org/}{\LaTeX}
\item \href{https://markdown.es/}{Markdown}
\item \href{http://www.stack.nl/\%7Edimitri/doxygen/index.html}{Doxygen}
\item \href{http://mtmacdonald.github.io/docgen/docs/index.html}{DocGen}
\item \href{http://pandoc.org/}{Pandoc}
\end{itemize}
\end{multicols}



\subsubsection{Gestión y planificación de proyectos}\index{planificación}
\begin{multicols}{2}
\begin{itemize}
\item \href{https://trello.com/}{Trello}
\item \href{https://es.atlassian.com/software/jira}{Jira}
\item \href{https://asana.com/}{Asana}
\item \href{https://slack.com/}{Slack}
\item \href{https://basecamp.com/}{Basecamp}
\item \href{https://www.teamwork.com/project-management-software}{Teamwork Projects}
\item \href{https://www.zoho.com/projects/}{Zoho Projects}
\end{itemize}
\end{multicols}


\subsection{Fuentes de información adicional}
\begin{itemize}
\item \href{https://leankit.com/blog/2019/03/top-6-software-development-methodologies/}{Top
  6 Software Development Methodologies}. Maja Majewski. Planview
  LeanKit, 2019.
\item \href{https://acodez.in/12-best-software-development-methodologies-pros-cons/}{12
  Best software development methodologies with pros and cons}. acodez,
  2018.
\item \href{http://www.itinfo.am/eng/software-development-methodologies/}{Software
  Development Methodologies}. Association of Modern Technologies
  Professionals, 2019.
\end{itemize}

\chapter{Resultados}
\label{cap:Resultados}

En esta sección se describirá la aplicación del método de trabajo presentado en el capítulo \ref{cap:Metodologia},  mostrando los elementos (modelos, diagramas, especificaciones, etc.) más importantes. 

Este apartado debe explicar cómo el empleo de la metodología permite satisfacer tanto el objetivo principal como los específicos planteados en el TFG así como los requisitos exigidos (según exposición en cap.~\ref{cap:Objetivo}).
\chapter{Conclusiones}
\label{cap:Conclusiones}

En este capítulo se realizará un juicio crítico y discusión sobre los resultados obtenidos. \emph{Cuidado, esta discusión no debe confundirse con una valoración del enriquecimiento personal que supone la realización del trabajo como culminación de una etapa académica}. Aunque de gran importancia, esta última valoración debe quedar fuera de la memoria del trabajo y solo debe ahondarse en ella ante requerimiento explícito del comité en el acto de defensa.

Si es pertinente deberá incluir información sobre trabajos derivados como publicaciones o ponencias en preparación, así como trabajos futuros \emph{(solo si estos están iniciados o planificados en el momento que se redacta el texto)}. Evitar hacer una lista de posibles mejoras. Contrariamente a lo que alguno pueda pensar generalmente aportan impresión de trabajo incompleto o inacabado.\footnote{Puede reflexionarse en ello por si en la defensa del trabajo se pregunta sobre estas posibles mejoras.}


\section{Justificación de competencias adquiridas}
Es muy importante recordar que según la normativa vigente, el capítulo de conclusiones debe incluir \emph{obligatoriamente} un apartado destinado a justificar la aplicación en el TFG de competencias específicas (dos o más) asociadas a la tecnología específica cursada.\index{competencias@\textbf{competencias}}

En el TFG se han trabajado las competencias correspondientes a la Tecnología Específica de \emph{[poner lo que corresponda]}:

\begin{description}
\item[Código de la competencia 1:] \emph{[Texto de la competencia 1]}. Explicación de cómo dicha competencia se ha trabajado en el TFG.
\item[Código de la competencia 2:] \emph{[Texto de la competencia 2]}. Explicación de cómo dicha competencia se ha trabajado en el TFG.
\end{description}

\dots otras más si las hubiera.


\section{Planificación y costes}
En este capítulo se puede incluir una valoración del trabajo realizado en el que se justifique el tiempo dedicado al TFG teniendo en cuenta que este tiene asignados 12 créditos ECTS\index{ECTS} que se traducen en {300-360} horas totales. En este sentido una correcta planificación del TFG debería garantizar que el trabajo se realice dentro de la horquilla señalada. Queda a criterio del tribunal la evaluación de valores extremos por defecto o exceso en función de los resultados obtenidos.

Es muy importante que todas las justificaciones aportadas se sustenten no solo en juicios de valor sino en evidencias tangibles como: historiales de actividad, repositorios de código y documentación, porciones de código, trazas de ejecución, capturas de pantalla, demos, etc.




% -------------------------

% No olvides retornar al interlineado sencillo en el resto del documento.
\singlespacing

%--- BACKMATTER
%\backmatter (se comenta para que los apéndices puedan aparecer después de la bibliografía)


% -------------------------
%
%--- BIBLIOGRAFÍA
%
% -------------------------
% BEGIN_FOLD
\cleardoublepage % Necesario para ajustar el avance de página
\phantomsection  % Ojo necesario con hyperref.
\addcontentsline{toc}{chapter}{\bibname} % Añade la bibliografía al Índice de contenidos. Se debe mantaner cuando se desean subbibliografías.
%---
% Opción 1: Bibliografía con todas las fuentes en un apartado.
%---
\printbibliography
%---

%---
% Opción 2: Bibliografía con secciones separadas.
%---
%\printbibheading
% Se puede incluir un apartado de fuentes de consulta no citadas
% Como no están citadas es preciso incluir comando \nocite{<key>}
% Se añaden palabras clave a las entradas en el fichero *.bib para poder diferenciarlas
%\printbibliography[heading=subbibliography,keyword=consulta,title={Fuentes de consulta}]
%\printbibliography[heading=subbibliography,keyword=url,title={Direcciones de Internet}]
% -------------------------
% END_FOLD



% -------------------------
%
%--- ANEXOS: Comentar si no se desean incluir. [OPT.]
% Mover si se desea que aparezcan antes de la bobliografía.
%
% -------------------------
%BEGIN_FOLD
\appendix
\portadaAnexos % OPT. Añade una portada para anexos

% Tras este punto los capítulos se numeran con letras.
% Aquí todos los apéndices necesarios
\chapter{El primer anexo}
\label{cap:AnexoA}

Los anexos se incluirá de modo opcional material suplementario que podrá consistir en breves manuales, listados de código fuente, esquemas, planos, etc. Se recomienda que no sean excesivamente voluminosos, aunque su extensión no estará sometida a regulación por afectar esta únicamente al texto principal. 

\paragraph{Bibliografía}\index{bibliografía}
Esta sección, que si se prefiere puede titularse 
«Referencias»,\index{referencias} incluirá un listado por orden alfabético 
(primer apellido del primer autor) con todas las obras en que se ha basado 
para la realización del TFG\index{TFG} en las que se especificará: autor/es, 
título, editorial y año de publicación. Solo se incluirán en esta sección las 
referencias bibliográficas que hayan sido citadas en el documento. Todas las 
fuentes consultadas no citadas en el documento deberían incluirse en una 
sección opcional denominada <<Material de consulta>>, aunque preferiblemente 
estas deberían incluirse como referencias en notas a pie de página a lo largo 
del documento.

Se usará método de citación numérico con el número de la referencia empleada entre corchetes. La cita podrá incluir el número de página concreto de la referencia que desea citarse. Debe tenerse en cuenta que el uso correcto de la citación implica que debe quedar claro para el lector cuál es el texto, material o idea citado. Las obras referenciadas sin mención explícita o implícita al material concreto citado deberían considerarse material de consulta y por tanto ser agrupados como «Material de consulta» distinguiéndolas claramente de aquellas otras en las que si se recurre a la citación.

Cuando se desee incluir referencias a páginas genéricas de la Web sin mención expresa a un artículo con título y autor definido, dichas referencias podrán hacerse como notas al pie de página o como un apartado dedicado a las «Direcciones de Internet».

Todo el material ajeno deberá ser citado convenientemente sin contravenir los términos de las licencias de uso y distribución de dicho material. Esto se extiende al uso de diagramas y fotografías. El incumplimiento de la legislación vigente en materia de protección de la propiedad intelectual es responsabilidad exclusiva del autor independientemente de la cesión de derechos que este haya convenido.

%De este modo será responsable legal ante cualquier acción judicial derivada del incumplimiento de los preceptos aplicables. Así mismo ante dicha circunstancia los órganos académicos se reservan el derecho a imponer al autor la sanción administrativa que se estime pertinente. 

\paragraph{Índice Temático}\index{Indice Temático@Índice Temático}
Este índice es opcional y se empleará como índice para encontrar los temas tratados en el trabajo. Se organizará de modo alfabético indicando el número de página(s) en el que se aborda el tema concreto señalado.
 % Apéndice A (opcionales)

%---








% -------------------------
%
% OPT.: ÍNDICE TEMÁTICO: Comentar si no se desean incluir.
%
% -------------------------
% CONSEJO: Incluir los comandos mientras se escribe cada capítulo ya que hacerlo al final resulta tedioso.
%\cleardoublepage % Necesario para ajustar el avance de página
%\phantomsection  % Ojo necesario con hyperref.
%% Permite cambiar el nombre del índice temático
%%\renewcommand{\indexname}{Título del índice}
%\addcontentsline{toc}{chapter}{\indexname} % Añade al Índice de contenidos.
%\printindex  % Facilitado por makeidx (opcional, si no se usa no se imprime)
%---
%END_FOLD
\end{document}
