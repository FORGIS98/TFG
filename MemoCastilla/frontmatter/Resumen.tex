%--- Ajustes del documento.
\pagestyle{plain}	% Páginas sólo con numeración inferior al pie

% -------------------------
%
% RESUMEN:
% OJO: Si es preciso cambiar manualmente orden Resumen <-> Abstract
%
% -------------------------
%--- Resumen en español
\selectlanguage{spanish} % Selección de idioma del resumen.
\cleardoublepage % Se incluye para modificar el contador de página antes de añadir bookmark
\phantomsection  % Necesario con hyperref
\addcontentsline{toc}{chapter}{Resumen} % Añade al TOC.

\begin{abstract}
% EDITAR: Resumen (máx. 1 pág.)
\begin{center}
\emph{[... versión del resumen en español ...]}
\end{center}
En una página como máximo, el resumen explicará de modo breve la problemática que trata de resolver el trabajo \emph{(el `qué')}, la metodología para  abordar su solución  \emph{(el `cómo')} y los resultados obtenidos. En los trabajos cuyo idioma principal sea el inglés, el orden de \textsf{Resumen} y \textsf{Abstract} se invertirá.
\end{abstract}
%---



%--- Resumen en inglés
% Abstract
\selectlanguage{english} % Selección de idioma del resumen.
\cleardoublepage
\phantomsection % Necesario con hyperref
\addcontentsline{toc}{chapter}{Abstract} % Añade al TOC.

\begin{abstract}
% EDITAR: Abstract (máx. 1 pág.)

\begin{center}
\emph{[... english version of the abstract ...]}
\end{center}
English version for the abstract.	
\end{abstract}
%---

%--- Ajuste del idioma para el resto del documento.
\ifspanish
	\selectlanguage{spanish}% Emplea idioma español
\else
	\selectlanguage{english}% Emplea idioma inglés
\fi
