% -------------------------
%
% -NOTACIÓN: Lista de símbolos con significado especial.
%
% -------------------------
\cleardoublepage
\phantomsection % Necesario con hyperref

% El método mostrado en este fichero es un modo rápido de incluir nomeclatura y listade acrónimos. En trabajos donde se precise un trabajo más depurado e intensivo puede recurrirse a los paquetes:
%   - nomencl
%   - acronym

\chapter*{Notación y acrónimos} % Opción con * para que no aparezca en TOC ni numerada
\addcontentsline{toc}{chapter}{Notación y acrónimos} % Añade al TOC.

\section*{Notacion}
Ejemplo de lista con notación (o nomenclatura) empleada en la memoria del TFG.\footnote{Se incluye únicamente con propósito de ilustración, ya que el documento no emplea la notación aquí mostrada.}

\begin{tabular}{r r p{0.8\linewidth}}
$A, B, C, D$	& : & Variables lógicas \\
$f, g, h$		& :	& Funciones lógicas \\
$\cdot$			& : & Producto lógico (AND), a menudo se omitirá como en $A 
B$ en lugar de $A \cdot B$\\
$+$				& : & Suma aritmética o lógica (OR) dependiendo del 
contexto\\
$\oplus$		& : & OR exclusivo (XOR)\\
$\overline{A}$ o ${A}'$	& : & Operador NOT o negación
\end{tabular}

\section*{Lista de acrónimos}
Ejemplo de lista con los acrónimos empleados en el texto.

\begin{tabular}{r r p{0.8\linewidth}}
CASE& : &Computer-Aided Software Engineering \\
CTAN& : &Comprenhensive \TeX{} Archive network \\
IDE& : &Integrated Development Environment \\
ECTS& : &European Credit Transfer and Accumulation System \\
OOD& : &Object-Oriented Design \\
PhD& : &Philosophiae Doctor \\
RAD& : &Rapid Application Development \\
SDLC& : &Software Development Life Cycle \\
SSADM& : &Structured Systems Analysis \& Design Method \\
TFE& : &Trabajo Fin de Estudios \\
TFG& : &Trabajo Fin de Grado \\
TFM& : &Trabajo Fin de Máster \\
UML& : &Unified Modeling Language
\end{tabular}